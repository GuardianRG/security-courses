\documentclass[Screen16to9,17pt]{foils}
\usepackage{zencurity-slides}


% It-sikkerhedsupdate 2020
% Få fremtidssikret it-sikkerhedsstrategien. Arrangement for medlemmer af Forsikringsforbundet og PROSA.

% Hvad skal en ansvarlig it-sikkerhedsstrategi være for 2020. Hvilke emner er de vigtigste, og hvad er truslerne, hvis man ikke straks kommer i gang med de 10 vigtigste punkter.

% Foredraget er en gennemgang af de 10 vigtigste områder og emner, som en organisation skal have styr på i 2020, med referencer til aktuelle sager som eksempel.

% Punkterne vil inkludere de sædvanlige, kedelige, men nødvendige; backup, CMDB, brugerstyring, logging m.fl. - men med forslag til praktiske værktøjer for at understøtte dem hurtigt.

\begin{document}
\selectlanguage{danish}
\mytitlepage{It-sikkerhedsupdate}{2020}


\vskip 1cm
\centerline{\footnotesize slides are available on Github}

\slide{Goal for today}

\hlkimage{5cm}{Shaking-hands_web.jpg}

What are the things on the table for a responsible it-security strategy. Which subjects are most important, and what are the threats, if you dont get started immediately with the top 10 priorities.


\begin{list2}
\item Plan:
\item Approx 2h including break
\item Inspiration for solving the tasks, prioritizing the tasks
\item I dont have tailer made solutions or easy answers for your organisation
\item Less presentation, more dialog
\end{list2}


\slide{Happy New Year}

\hlkrightpic{10cm}{0cm}{happy-new-year-roven-images-601197-unsplash.jpg}
{~}

\begin{list2}
\item Same problems
\item Repeat last year?
\item ... or try something new!
\item 2020 was a nightmare of break-ins and data leaks
\item GDPR is here and the snow ball is rolling
\end{list2}

\vskip 1cm
{\LARGE\bf Try not to panic, but there are lots of threats}


\slide{Paranoia defined}

\hlkimage{12cm}{paranoia-definition.png}

Source: google paranoia definition


\slide{Hackers don't give a shit}

\hlkrightpic{11cm}{-3cm}{kiwicon-2009-hackers-dont-give-shit.jpg}

Your system is only for testing, development, ...

Your network is a research network, under construction, \\
being phased out, ...

Try something new, go to your management

Bring all the exceptions, all of them, update the risk \\
analysis figures - if this happens it is about 1mill DKK

Ask for permission to go full monty on your security

{\bf Think like attackers - don't hold back}


\slide{Confidentiality Integrity Availability}

\hlkimage{8cm}{cia-triad-uk.pdf}

\begin{list1}
\item We want to protect something
\item Confidentiality - data is kept a secret
\item Integrity - data is not subjected to unauthorized changes
\item Availability - data and systems are available for authorized users when they need them
\end{list1}

\slide{Security engineering a job role}

\begin{alltt}\small
On any given day, you may be challenged to:
        Create new ways to solve existing production security issues
        Configure and install firewalls and intrusion detection systems
        Perform vulnerability testing, risk analyses and security assessments
        Develop automation scripts to handle and track incidents
        Investigate intrusion incidents, conduct forensic investigations and incident responses
        Collaborate with colleagues on authentication, authorization and encryption solutions
        Evaluate new technologies and processes that enhance security capabilities
        Test security solutions using industry standard analysis criteria
        Deliver technical reports and formal papers on test findings
        Respond to information security issues during each stage of a project’s lifecycle
        Supervise changes in software, hardware, facilities, telecommunications and user needs
        Define, implement and maintain corporate security policies
        Analyze and advise on new security technologies and program conformance
        Recommend modifications in legal, technical and regulatory areas that affect IT security
\end{alltt}

Source: \url{https://www.cyberdegrees.org/jobs/security-engineer/}\\
also
\url{https://en.wikipedia.org/wiki/Security_engineering}



\slide{Focus 2020}

\begin{list2}
\item User management - including administrative users
\item Asset management
\item Laptop security
\item VPN everywhere
\item Penetration testing
\item Firewalls and segmentation
\item TLS and VPN settings, encryption
\item DNS and email security
\item Syslog and monitorering
\item Incident Response and response
\end{list2}

\vskip 5mm
\centerline{I hope you have a team, otherwise choose a few at a time}

\slide{Focus 2020: User management}

\hlkimage{8cm}{humans2.png}

\begin{list2}
\item Relevant for alle organisationer
\item Er måden vi sikrer godkendte brugere kan udføre opgaver
\item Kodeord bruges til at forhindre uautoriseret adgang
\item Har I styr på brugerid?
\item Hvor er brugere oprettet?
\item Hvor hurtigt kan I fjerne "een bruger" eller "deaktivere en bruger" alle steder!
\item Er det et kludetæppe - ja, mange steder er det
\end{list2}


\slide{Centraliseret brugerstyring}

\begin{list1}
\item Active Directory, mange danske virksomheder bruger det
\item LDAP central brugerstyring
\item ... men brug det endnu mere
\begin{list2}
\item Konfigurer applikationer til central styring
\item Fjern applikationer som ikke tillader central styring
\item Overvågning på fejlslagne logins, og godkendte logins
\end{list2}
\item Generelt minimer brugere andre steder end i den centrale database
\end{list1}

\vskip 1cm
Hvad med ILO, DRAC, temperaturovervågning - en fælles password database, med begrænset adgang, måske?


\slide{Passwords vælges ikke tilfældigt}

\hlkimage{20cm}{50-most-used-passwords.png}

Source:
\link{https://wpengine.com/unmasked/}


\slide{Your data has already have been owned by criminals}

\hlkimage{13cm}{pwned.png}

\begin{list1}
\item Your data is already being sold, and resold on the Internet
\item Stop reusing passwords, use a password safe to generate and remember
\item Check you own email addresses on \link{https://haveibeenpwned.com/}
\end{list1}

\centerline{They have an API you can integrate to}


\slide{Brug mere sikre passwords}

\begin{quote}
Pwned Passwords overview\\
Pwned Passwords are more than half a billion passwords which have previously been exposed in data breaches. The service is detailed in the launch blog post then further expanded on with the release of version 2. The entire data set is both downloadable and searchable online via the Pwned Passwords page.
\end{quote}

\begin{list1}
\item I kan forhindre brugere i at vælge passwords der ALLEREDE er lækket
\item I kan bruge deres API eller download\\
{\footnotesize\link{https://www.troyhunt.com/introducing-306-million-freely-downloadable-pwned-passwords/}}
\end{list1}


\slide{Focus 2020: Asset management}

\hlkimage{7cm}{old_book_lumen_design_st_01.png}

\begin{list2}
\item Specielt relevant for mellemstore til store organisationer
\item Hvilke assets har vi?
\item Hvordan sikrer vi at vi ikke mister værdierne
\end{list2}


\slide{Hvad er asset management}

\begin{quote}
CIS Control 1:\\
Inventory and Control of Hardware Assets
Actively manage (inventory, track, and correct) all hardware devices on the network so that only
authorized devices are given access, and unauthorized and unmanaged devices are found and
prevented from gaining access.
\end{quote}
Source: \link{https://www.cisecurity.org/}

\begin{list2}
\item Hardware - både indkøbte, opkoblede, udlånte, stjålne ...
\item Software - licenser, indkøb, brug, opgraderingspriser
\item Virtuelle arkiver - eksempelvis forretningskritiske data
\item ...
\end{list2}

\slide{Hardware asset management}

\hlkimage{10cm}{racktables-shot-indexpage.png}

\begin{list2}
\item Der findes mange systermer
\item Det anbefales at bruge specialiserede systemer, a la RackTables:\\
Have a list of all devices you've got,
Have a list of all racks and enclosures,
Mount the devices into the racks,
Maintain physical ports of the devices and links between them
\end{list2}

\slide{Software asset management - virtuelle arkiver}

\hlkimage{9cm}{datalaek-2019.png}

\begin{list2}
\item Software - licenser, indkøb, brug, opgraderingspriser
\item Virtuelle maskiner - er en server et asset, eller er det data?
\item IP adresser
\item Data arkiver - GDPR
\end{list2}

\slide{Focus 2020: Laptop sikkerhed}

\hlkimage{13cm}{kelly-sikkema-212376-unsplash.jpg}

\begin{list2}
\item Relevant for alle
\item Hvordan sikrer vi at vi ikke mister værdierne, hardware og data typisk
\end{list2}


\slide{Secure Laptops}

\hlkimage{10cm}{librem-15-v3-turns99.png}

\begin{list2}
\item Laptops (og mobile enheder)
\item Hvad kendetegner en laptop?
\item Hardware naturligvis, en Macbook koster officielt mere end en brugt mellemklassebil
\item - og husk brugen af laptops, må der downloades data til offline
\item Er laptops sikre, og hvad betyder det?
\end{list2}



\slide{Are your data secure - data at rest}

\hlkimage{15cm}{images/data-integrity-1.pdf}

\begin{list1}
\item Stolen laptop, tablet, phone - can anybody read your data?
\item Do you trust "remote wipe"
\item How do you in fact wipe data securely off devices, and SSDs?
\item Encrypt disk and storage devices before using them in the first place!
\end{list1}

\slide{2018 attack}

\hlkimage{12cm}{ssd-attack-2018.png}
\emph{self-encrypting deception: weakness in the encryption of solid state drives (SSDs)}\\
\link{https://www.ru.nl/publish/pages/909282/draft-paper.pdf}





\slide{Recommendations - Comply Everywhere, Act Anywhere}

\hlkrightpic{5cm}{-1cm}{003scawebgoshindomanicon.png}
{~}
\begin{list1}
\item {\bf Laptop storage must be encrypted}
\item Firewall must be enabled
\item Next suggestion:
\begin{list2}
\item Try sniffing data from a laptop, setup Access Point/Monitor port
\item Portscan your laptop - use Nmap
\item Write an email to everyone in your organisation:\\
"Hi All, we need to identify laptops without full disk encryption \\
- come see us, we have christmas cookies left, Best regards IT"
\end{list2}
\end{list1}


\slide{Focus 2020: Penetration testing}

\hlkimage{19cm}{burp-default-proxy-intercept.png}

\begin{list2}
\item Relevant hvis du driver et netværk, specielt hvis det er forbundet til internet eller stort
\item Du bliver hele tiden testet - internet-tinnitus
\item Penetration testing
\item Kontrol af sikkerheden med aktive værktøjer
\item Brug Nmap pakken til at checke åbne porte
\item Køb Burp Suite hvis du har et web site du tjener penge på
\end{list2}


\slide{Start Attacking from the Inside}

\hlkimage{6cm}{erik-odiin-568459-unsplash.jpg}

\begin{list2}
\item Now imagine you were in control of a company laptop
\item Do you have a large internal world wide network?\\
Having a large open network may cost you 1.9 billion DKK - ref Maersk
\item Try scanning everything, start in a small corner, expand
\item Scan all you danish segments, one by one, then the nordic, then the world
\item Yes, things may break - FINE, BREAKING is GOOD
\end{list2}

\centerline{\bf Better to break while we are ready to un-break}


\slide{How to break stuff}

Think like an attacker, and begin at the bottom.

I sit here, but where am I connected:
\begin{alltt}\footnotesize
reading from file cisco-lldp-1.cap, link-type EN10MB (Ethernet)
16:39:43.468745 LLDP, length 328
        Chassis ID TLV (1), length 7
          Subtype MAC address (4): 70:ff:1a:01:03:02 (oui Unknown)
        Port ID TLV (2), length 8
          Subtype Local (7): Eth1/47
        Port Description TLV (4), length 12: Ethernet1/47
        System Description TLV (6), length 158
          Cisco Nexus Operating System (NX-OS) Software 14.0(2c) TAC support: http://www.cisco.com/tac Copyright (c) 2002-2020, Cisco Systems, Inc. All rights reserved.
\end{alltt}

\vskip 5mm
\centerline{I love LLDP, but it does reveal software version, so which flaws available}

Check \emph{Security Assessment of Cisco ACI}\\
\link{https://www.ernw.de/en/whitepapers/issue-68.html}
\slide{Hackertools are for everyone!}

\hlkrightpic{10cm}{0cm}{kali-linux.png}
{~}

\begin{list2}
\item Hackers work all the time to break stuff, Use hackertools:
\item Nmap, Nping \link{http://nmap.org}
\item Wireshark - \link{http://www.wireshark.org/}
\item Aircrack-ng \link{http://www.aircrack-ng.org/}
\item Metasploit Framework \link{http://www.metasploit.com/}
\item Burpsuite \link{http://portswigger.net/burp/}
\item Kali Linux \link{http://www.kali.org}
\end{list2}

\centerline{Most popular hacker tools \link{http://sectools.org/}}


\slide{Really do Nmap your world}

\hlkrightpic{10cm}{0cm}{nmap-zenmap.png}
{~}

\begin{list2}
\item Nmap is a port scanner, but does more
\item Finding your own infrastructure available from the guest network?
\item See your printers having all the protocols enabled AND a wireless?
\item Includes scripting, and a lot of useful scripts by default
\item Often when a new vuln is published, there will be a test script for Nmap
\end{list2}


\slide{Hackerlab setup}

\hlkimage{11cm}{hacklab-1.png}

\begin{list2}
\item Create hacker labs, encourage hacker labs!
\item Software Host OS: Windows, Mac, Linux
\item Virtualisation software: VMware, Virtual box, HyperV pick your poison
\item Hackersoftware: Kali Virtual Machine \link{https://www.kali.org/}
\end{list2}



\slide{Focus 2020: Firewalls og segmentering}

\hlkimage{8cm}{virksomhedens-netvaerk.pdf}

\begin{list2}
\item Hvis du har et netværk, så bør du have en firewall
\item En firewall tillader autoriseret trafik og blokerer resten
\item Hvornår har du sidst set din løsning efter?
\item Hvor lang tid tager det at se en 5.000 linier Cisco ASA config igennem?
\item Segmentering af netværk er en solid sikkerhedsforanstaltning
\end{list2}

\slide{Big firewalls}

\hlkimage{14cm}{network-layers-1.png}

\centerline{Big firewalls are not a single device}

PS also check for updates to your network devices, at least once a year \smiley


\slide{Focus 2020: VPN alle steder}

\hlkimage{12cm}{ks-kyung-784757-unsplash.jpg}

\begin{list2}
\item VPN er relevant for alle der har data af værdi
\item Sikrer data der flyttes
\item Virtual Private Network dækker over klienter der kobler op, og site-2-site
\end{list2}

\slide{Your Privacy }

\hlkimage{15cm}{images/internet-browsing.pdf}


\begin{list2}
\item Your data travels far
\item Often crossing borders, virtually and literally
\end{list2}

\vskip 5mm
\centerline{\bf\Large Maybe use VPN more - or always!}


\slide{Focus 2020: TLS og VPN indstillinger}

\begin{alltt}\footnotesize
  # Input from https://github.com/tykling/ansible-roles/blob/master/nginx_server/templates/tls.conf.j2#L6
  ssl_protocols                   TLSv1.2 TLSv1.3;
  ssl_ciphers                     ECDHE-RSA-AES256-GCM-SHA384:ECDHE-RSA-CHACHA20-POLY1305:ECDHE-RSA-AES128-GCM-SHA256:ECDHE-RSA-AES256-SHA384:ECDHE-RSA-AES128-SHA256:ECDHE-RSA-AES256-SHA;
  ssl_prefer_server_ciphers       on;
  ssl_session_cache               shared:SSL:10m;     ssl_session_tickets       off;   ssl_session_timeout    4h;
  ssl_stapling                    on;                 ssl_stapling_verify       on;
  resolver                        105.238.53.1;
  ssl_ecdh_curve secp384r1;                           ssl_dhparam /etc/nginx/dh4096.pem;
  add_header Strict-Transport-Security "max-age=31536000; includeSubDomains" always;
  add_header Referrer-Policy "no-referrer";  add_header X-Content-Type-Options "nosniff";
  add_header X-Frame-Options "DENY";  add_header X-XSS-Protection "1; mode=block";
  add_header Content-Security-Policy "default-src 'self'; script-src 'self; img-src 'self'; object-src 'none'; font-src 'self'; frame-ancestors 'none' https:";
\end{alltt}

\begin{list2}
\item De fleste har https nu, men er det konfigureret optimalt
\item Anbefaler at alle indstillingerne gennemgås regelmæssigt!
\item Lav et dokument med de indstillinger I bruger i jeres organisation
\end{list2}

\slide{Nmap efter SSL og TLS}

\hlkimage{6cm}{nmap-sslv2.png}

\begin{list1}
\item Nu vi har lært Kali og Nmap at kende
\begin{list2}
\item Find nemt alle ssl version 2 og 3\\
\verb+nmap --script ssl-enum-ciphers+
\item Brug ssllabs https://www.ssllabs.com/ - kræver hostnavn og til HTTPS
\item sslscan kommandoen kan checke alle jeres TLS sites, også på IP
\end{list2}
\end{list1}


\slide{sslscan}

\begin{alltt}\small
root@kali:~# sslscan --ssl2 web.kramse.dk
Version: 1.10.5-static
OpenSSL 1.0.2e-dev xx XXX xxxx

Testing SSL server web.kramse.dk on port 443
...
  SSL Certificate:
Signature Algorithm: sha256WithRSAEncryption
RSA Key Strength:    2048
\end{alltt}

Source:
Originally sslscan from http://www.titania.co.uk
 but use the version on Kali

SSLscan can by IP and also SMTP STARTTLS and others

You can check many settings for a domain easily with 
\link{https://internet.nl/}


\slide{VPN indstillinger}

\hlkimage{9cm}{crypto-rot13.pdf}

\begin{list1}
\item PPTP, fjern, kill on sight
\item Check hvert år:
\begin{list2}
\item Certifikater/nøgler - ligesom TLS lange og rulles indimellem
\item Algoritmer DES/3DES, SHA1, MD5 bye bye, husk både encryption og auth
\item DH-Group - mindst +15 tak
\item Check både client VPN og site-2-site
\end{list2}
\end{list1}

\slide{Anbefalinger til VPN}

\begin{quote}
  Use the following guidelines when configuring Internet Key Exchange (IKE) in VPN technologies:\\
* Avoid IKE Groups 1, 2, and 5.\\
* Use IKE Group 15 or 16 and employ 3072-bit and 4096-bit DH, respectively.\\
* When possible, use IKE Group 19 or 20. They are the 256-bit and \\
384-bit ECDH groups, respectively.\\
* Use AES for encryption.
\end{quote}
Paper:\\
{\footnotesize \link{https://www.cisco.com/c/en/us/about/security-center/next-generation-cryptography.html}}

\begin{list2}
\item Certifikater/nøgler - ligesom TLS lange og rulles indimellem
\item Check både client VPN og site-2-site
\end{list2}

\slide{Focus 2020: DNS og email}

\hlkrightpic{8cm}{-2cm}{brian-patrick-tagalog-680954-unsplash.jpg}
{~}

\begin{list2}
\item Vi er afhængige af email, modtagelse og afsendelse
\item Når vi modtager skal det helst gå hurtigt
\item Når vi sender skal vi ikke ende i spam mappen
\item Phishing, hvem kan sende \emph{fra vores domæne}
\end{list2}


\slide{Various key attack types, clients and employees}

\begin{list2}
\item Phishing - sending fake emails, to collect credentials
\item Spear phishing - targetted attacks
\item Person in the middle - sniffing and changing data in transit
\item Drive-by attacks - web pages infected with malware, often ad servers
\item Malware transferred via USB or email
\item Credential Stuffing, Password related, like re-use of password, see slide about being pwned
\end{list2}

\vskip 1cm
\centerline{\Large\bf If we all wait a bit, and not click links immediately}

\vskip 1cm
Hackers try to create "urgency", click this or loose money


\slide{DNSSEC get started now}

\hlkimage{12cm}{cz-nic-dnssec-tlsa-validator.png}

\begin{quote}
"TLSA records store hashes of remote server TLS/SSL certificates. The authenticity of a TLS/SSL certificate for a domain name is verified by DANE protocol (RFC 6698). DNSSEC and TLSA validation results are displayer by using several icons."
\end{quote}


\slide{DNSSEC and DANE}

\begin{quote}
"Objective:

Specify mechanisms and techniques that allow Internet applications to
establish cryptographically secured communications by using information
distributed through DNSSEC for discovering and authenticating public
keys which are associated with a service located at a domain name."
\end{quote}

\begin{list1}
\item DNS-based Authentication of Named Entities (dane)
\end{list1}


\slide{Email security 2020 - Goals}

\begin{list2}
\item SPF Sender Policy Framework\\ {\footnotesize\link{https://en.wikipedia.org/wiki/Sender_Policy_Framework}}
\item DKIM DomainKeys Identified Mail\\
{\footnotesize\link{https://en.wikipedia.org/wiki/DomainKeys_Identified_Mail}}
\item DMARC Domain-based Message Authentication, Reporting and Conformance\\
{\footnotesize\link{https://en.wikipedia.org/wiki/DMARC}}
\item DNSSEC Domain Name System Security Extensions\\ {\footnotesize\link{https://en.wikipedia.org/wiki/Domain_Name_System_Security_Extensions}}
\item DANE DNS-based Authentication of Named Entities\\ {\footnotesize\link{https://en.wikipedia.org/wiki/DNS-based_Authentication_of_Named_Entities}}
\item Brug allesammen, check efter ændringer!
\end{list2}

\centerline{Jeg er glad for at teste med \link{https://dmarcian.com/}}

\slide{Focus 2020: Syslog og monitorering}

\hlkimage{7cm}{lup.png}

\begin{list2}
\item Vi har allesammen security incidents
\item Vi skal kunne efterforske, derfor er et niveau af syslog vigtigt
\item Også i dagligdagen til at sikre at systemerne kører optimalt
\end{list2}

\slide{Network tools - examples}

\hlkimage{10cm}{kibana-solido.png}

\begin{list2}
\item Net + SSL/TLS: Zeek \link{http://www.bro-ids.org} \\Suricata \link{http://suricata-ids.org}
\item DNS query logs, keep it for at least a week?\\
- with DSC and PacketQ \link{https://github.com/DNS-OARC/PacketQ}
\item Log with Elasticsearch?\\
{\footnotesize\link{https://www.elastic.co/guide/en/elasticsearch/guide/current/index.html}}
\end{list2}

\slide{Network visibility: Netflow with NFSen}

\hlkimage{15cm}{nfsen-udp-flood.png}

\centerline{An extra 100k packets per second from this netflow source (source is a router)}


\slide{Case: Maltrail}

\hlkimage{15cm}{maltrail.png}

\link{https://github.com/stamparm/maltrail}



% Suricata, Logstash, Elasticsearch, D3JShttp://d3js.org/
\slide{Suricata with Dashboards}

\hlkimage{12cm}{kibana-suricata.png}

Picture from Twitter\\
\link{https://twitter.com/nullthreat/status/445969209840128000}\\



\slide{How to get started}

\begin{list1}
\item How to get started searching for security events?
\item Collect basic data from your devices and networks

\begin{list2}
\item Netflow data from routers
\item Session data from firewalls
\item Logging from applications: email, web, proxy systems
\end{list2}
\item {\bf Centralize!}
\item Process data
\begin{list2}
\item Top 10: interesting due to high frequency, occurs often, brute-force attacks
\item {\it ignore}
\item Bottom 10: least-frequent messages are interesting
\end{list2}
\end{list1}

\slide{Reklame: SIEM og log-analyse  (5 ECTS)}

Teaching dates: {\bf 26/11 2020}, 1/12 2020, 3/12 2020, 8/12 2020, 10/12 2020, 15/12 2020, 17/12 2020
Exam: Date 7/1 2021

Primary literature:
\begin{list2}
\item Data-Driven Security: Analysis, Visualization and Dashboards Jay Jacobs, Bob Rudis ISBN: 978-1-118-79372-5 February 2014 https://datadrivensecurity.info/
\item Crafting the InfoSec Playbook: Security Monitoring and Incident Response Master Plan by Jeff Bollinger, Brandon Enright, and Matthew Valites
\item Intelligence-Driven Incident Response ISBN: 9781491934944 Scott Roberts
\item Security Operations Center Building, Operating, and Maintaining your SOC ISBN: 9780134052014 Joseph Muniz
\end{list2}

{\footnotesize
\link{https://kompetence.kea.dk/kurser-fag/siem-og-loganalyse?kust_id=5154}

\link{https://zencurity.gitbook.io/kea-it-sikkerhed/siem-and-log-analysis/lektionsplan}}

\slide{Focus 2020: Incident Response og reaktion}

\hlkimage{10cm}{margarida-csilva-121801-unsplash.jpg}

\begin{list2}
\item Fortsat fra logningen ... hvad så nu!
\item Hvis du har en sikkerhedshændelse skal den håndteres
\item jo hurtigere og mere effektivt det håndteres jo bedre
\end{list2}

Lifeguard training photo by Margarida CSilva on Unsplash

\slide{Overlapping Security Incidents}

\hlkrightpic{12cm}{1cm}{datalaek-2019.png}

New data breaches nearly every week, these from danish news site \link{version2.dk}

Problem, we need to receive data from others

Data from others may contain malware

Have a job posting, yes\\
- then HR will be expecting CVs sent as .doc files

\slide{}

or the other way

{\Large\bf Attackers used a LinkedIn job ad\\
and Skype call to breach bank’s defences}
\hlkimage{12cm}{redbanc-skype-malware.png}

{\footnotesize
\link{https://nakedsecurity.sophos.com/2020/01/21/attackers-used-a-linkedin-job-ad-and-skype-call-to-breach-banks-defences/}}

\slide{Øv krisesituationer}

\hlkimage{12cm}{sheldon-nunes-1226991-unsplash.jpg}
\begin{list2}
\item Lav rollespil
\item Lav tabletop exercises
\end{list2}

\slide{Spørgsmål og mere debat}

\hlkimage{7cm}{idog.jpg}

\begin{center}
\hlkbig

\myname

\end{center}




\end{document}
