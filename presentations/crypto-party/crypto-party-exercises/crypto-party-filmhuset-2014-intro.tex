\documentclass[20pt,a4paper,footrule]{report}
\input{zencurityreport.tex}
\usepackage{pdf14}
%\usepackage{ulem}

% Basic things that we need are below
\selectlanguage{danish}
\lhead{Crypto party for teens (and others)}
\rhead{Social Media Week Copenhagen}
%\lfoot{\color{black}Copyright \copyright\ 2014 Solido Networks\\
%Permission to use, copy and distribute this for any purpose
%is hereby granted.}
\cfoot{}

\begin{document}
\normal

\section*{CryptoParty vejledning}

CryptoParty er en mulighed for inspiration til bedre sikkerhed. Vi anbefaler at du kigger p� programmerne nedenfor. Du kan l�se mere om Crypto Parties i The CryptoParty Handbook\\
\link{https://www.cryptoparty.in/documentation/handbook}
\hlkimage{4cm}{crypto-party-logo.png}

NB: The CryptoParty Handbook indeholder referencer til flere programmer s�som disk kryptering og VPN

\section*{Nemt og hurtigt: flere browsere}

\hlkimage{10cm}{multi-browser-strategy.png}

\begin{itemize}
\item Flere browsere: en til facebook, en til netbank,\\
eksempelvis Firefox til generel surfing og Safari til netbanken
\item Sikre indstillinger og NoScripts til generel surfing
\item L�sere indstillinger til Netflix, Facebook m.fl.
\item Husk at installere gode plugins som HTTPS Everywhere \link{https://www.eff.org/Https-everywhere}, NoScript, CertPatrol m.fl.
\item Mere anonym browsing kan opn�s med Tor fra \link{torproject.org} og \link{https://www.whonix.org/}
\end{itemize}

\section*{Sikker email med Thunderbird, OpenPGP og Enigmail}

\hlkimage{8cm}{thunderbird-enigmail-logo.png}
\begin{itemize}
\item Hent programmerne fra \link{http://www.mozilla.org/en-US/thunderbird/} og \link{https://www.enigmail.net/}
\item F�lg vejledningerne p� hjemmesiden \link{https://www.enigmail.net/documentation/quickstart.php}
\item GnuPG programmet bruges til at kryptere med \link{http://www.gnupg.org/}
\end{itemize}



\end{document}
