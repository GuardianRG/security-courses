\documentclass[a4paper,11pt,notitlepage]{report}
% Henrik Lund Kramshoej, February 2001
% hlk@security6.net,
% My standard packages
\usepackage{zencurity-network-exercises}

\begin{document}

\rm
\selectlanguage{english}


\newcommand{\emne}[1]{Sikker browsing, plugins og Tor project}
\newcommand{\kursus}[1]{\emne{} workshop}
\newcommand{\kursusnavn}[1]{\emne\\ exercises}

\mytitle{\emne}{workshop}


\thispagestyle{empty}
\eject

\pagenumbering{roman}
\setlength{\parskip}{0pt}

\setcounter{tocdepth}{0}

\normal

{\color{titlecolor}\tableofcontents}
%\listoffigures - not used
%\listoftables - not used

\normal
\pagestyle{fancyplain}
\chapter*{\color{titlecolor}Preface}
\markboth{Preface}{}

This material is prepared for use in \emph{\kursus} and was prepared by
Henrik Lund Kramshoej, \link{http://www.zencurity.com} .
It describes the networking setup and
applications for trainings and workshops where hands-on exercises are needed.

\vskip 1cm
Further a presentation is used which is available as PDF from kramse@Github\\
Look for \jobname in the repo security-courses.

These exercises are expected to be performed in a training setting with network connected systems. The exercises use a number of tools which can be copied and reused after training. A lot is described about setting up your workstation in the repo

\link{https://github.com/kramse/kramse-labs}



\section*{\color{titlecolor}Prerequisites}

This material expect that participants have a working knowledge of
TCP/IP from a user perspective. Basic concepts such as web site addresses and email should be known as well as IP-addresses and common protocols like DHCP.

\vskip 1cm
Have fun and learn
\eject

% =================== body of the document ===============
% Arabic page numbers
\pagenumbering{arabic}
\rhead{\fancyplain{}{\bf \chaptername\ \thechapter}}

% Main chapters
%---------------------------------------------------------------------
% gennemgang af emnet
% check questions


\chapter*{\color{titlecolor}Introduction to networking}
%\markboth{Introduktion til netværk}{}
\label{chap:intro}

\section*{\color{titlecolor}IP - Internet protocol suite}

It is extremely important to have a working knowledge about IP to implement
secure and robust infrastructures. Knowing about the alternatives while doing
implementation will allow the selection of the best features.

\section*{\color{titlecolor}ISO/OSI reference model}
A very famous model used for describing networking is the ISO/OSI model
of networking which describes layering of network protocols in stacks.

This model divides the problem of communicating into layers which can
then solve the problem as smaller individual problems and the solution
later combined to provide networking.

Having layering has proven also in real life to be helpful, for instance
replacing older hardware technologies with new and more efficient technologies
without changing the upper layers.

In the picture the OSI reference model is shown along side with
the Internet Protocol suite model which can also be considered to have different layers.


\begin{figure}[H]
\label{fig:osi}
\begin{center}
\colorbox{white}{\includegraphics[width=8cm,angle=90]{images/compare-osi-ip.pdf}}
\end{center}
\caption{OSI og Internet Protocol suite}
\end{figure}


\chapter*{\color{titlecolor}Exercise content}
\markboth{Exercise content}{}

Most exercises follow the same procedure and has the following content:
\begin{itemize}
\item {\bf Objective:} What is the exercise about, the objective
\item {\bf Purpose:} What is to be the expected outcome and goal of doing this exercise
\item {\bf Suggested method:} suggest a way to get started
\item {\bf Hints:} one or more hints and tips or even description how to
do the actual exercises
\item {\bf Solution:} one possible solution is specified
\item {\bf Discussion:} Further things to note about the exercises, things to remember and discuss
\end{itemize}

Please note that the method and contents are similar to real life scenarios and does not detail every step of doing the exercises. Entering commands directly from a book only teaches typing, while the exercises are designed to help you become able to learn and actually research solutions.




\chapter{Installation af alternativ browser}
\label{ex:browser-install}

{\bf Objective:}\\
Installer en alternativ browser på din PC, eksempelvis Firefox eller Chrome

{\bf Suggested method:}\\
Hent installationsprogrammet fra \link{http://www.mozilla.org} eller \link{http://www.google.com/chrome}



En ekstra browser giver mulighed for nemt at have sikre indstillinger når man surfer på internet.

{\bf Solution:}\\
Hent installationsprogrammet og udfør installationen

{\bf Discussion:}\\
Vi bruger Firefox for at have en alternativ browser, som kan indstilles mere paranoidt, kan udvides med plugins, Flash blocker m.v.

Det er valgfrit hvilken browser man vælger, men en alternativ browser giver muligheder for bedre sikkerhedsindstillinger.

Mac brugere kan derefter bruge Safari, mens Windows brugere kan fortsætte med Internet Explorer til NemID/Netbank og sites man stoler på.

Husk at installere plugins som:
\begin{itemize}
\item Firefox CertPatrol
\item Firefox / Chrome - HTTPS Everywhere
\item Firefox NoScripts/ Chrome NotScripts
\end{itemize}


\chapter{Installation af Torbrowser}
\label{ex:torbrowser}

{\bf Objective:}\\
Installer Torbrowser pakken fra \link{http://www.torproject.org/}

{\bf Suggested method:}\\
Hent installationsprogrammet og installer

{\bf Suggested method:}\\

{\bf Solution:}\\

{\bf Discussion:}\\





\chapter{Installation af Thunderbird mail program}
\label{ex:thunderbird}



{\bf Objective:}\\
Installer Thunderbird mailklienten på din PC

{\bf Suggested method:}\\
Hent installationsprogrammet lokalt eller fra \link{http://www.mozilla.org}

{\bf Suggested method:}\\

{\bf Solution:}\\
Hent installationsprogrammet og udfør installationen.

{\bf Discussion:}\\
Thunderbird anbefales fremfor eksempelvis Outlook og Apple Mail grundet de
gode muligheder for udvidelser, herunder Enigmail OpenGPG plugin.

på Mac OS X kan benyttes den indbyggede Mail.app med GPGMail plugin - hvis man kan leve med at den ikke altid findes til nyeste version af Mac OS X.

på andre UNIX varianter er Mutt mail reader populær og integrerer nemt
til OpenGPG.

Thunderbird giver også mulighed for nem mail filtrering med eksempelvis Sieve og Dovecot.


\chapter{Installation af GPG GNU Privacy Guard}
\label{ex:gnupg}

{\bf Objective:}\\
Installer GNU Privacy Guard på jeres PC.

{\bf Suggested method:}\\
Hent installationsprogrammet og installer - brug pakkesystemerne hvis I bruger Linux

Mac OS X brugere kan med fordel benytte \link{https://www.gpgtools.org/}

{\bf Suggested method:}\\

{\bf Solution:}\\

{\bf Discussion:}\\

\chapter{Installation af Enigmail plugin}
\label{ex:enigmail}

{\bf Objective:}\\
Installer Enigmail plugin til Thunderbird

{\bf Suggested method:}\\
Hent installationsprogrammet og installer

{\bf Suggested method:}\\

{\bf Solution:}\\
Det nemmeste er at gå til hjemmesiden for Enigmail

\link{http://enigmail.mozdev.org/}

{\bf Discussion:}\\
Enigmail kræver at GNU Privacy Guard er installeret



\chapter{Lav en PGP-kompatibel nøgle}
\label{ex:generate-pgpkey}

{\bf Objective:}\\
Brug et valgfrit program til at lave en PGP-kompatibel nøgle

{\bfseries Forslag til fremgangsmåde:}\\
Brug Enigmail Key Manager eller PGP pakken til at generere en nøgle

{\bfseries Hjælp:}\\
Sørg for at lave din første nøgle med udløbsdato!

Sørg for at lave et revocation certificate på din første nøgle - så
kan du altid trøkke den tilbage - selvom du glemmer kodeordet.


{\bfseries Forslag til løsning:}\\
Brug det lokale mailsetup som eksempel og lav en nøgle


{\bf Discussion:}\\
Husk at hvis det skal være en rigtig nøgle skal du helst bruge din
rigtige mailadresse

Jeg vil ikke anbefale at der uploades "testnøgler" til keyservere, da
man ikke kan slette sine nøgler.


\chapter{Hent en nøgle fra en anden}
\label{ex:get-pgp-key}

{\bf Objective:}\\
Find en nøgle og indlæs den i din nøglering

{\bf Suggested method:}\\
Brug enten en keyserver \link{http://pgp.mit.edu} eller en USB nøgle
til at overføre en elektronisk udgave af nøglen til din PC

{\bf husk at verificere fingerprint}

{\bf Suggested method:}\\

{\bf Solution:}\\

{\bf Discussion:}\\


\chapter{Send en krypteret mail}
\label{ex:email-pgp-send}

{\bf Objective:}\\
Send en krypteret mail

{\bf Suggested method:}\\
Brug PGP pakken eller Thunderbird til at sende en krypteret mail til
en af de andre

{\bf Suggested method:}\\

{\bf Solution:}\\

{\bf Discussion:}\\




\chapter{Signer en nøgle}
\label{ex:pgp-sign-key}

{\bf Objective:}\\
Find ud af hvordan du laver en signatur på en nøgle og returnerer den

{\bf Suggested method:}\\


{\bf Suggested method:}\\

{\bf Solution:}\\

{\bf Discussion:}\\
Det er en god politik KUN at signere nøgler hvor man har fået fremvist
officielle papirer som kørekort og pas.

Læg yderligere mærke til at det som signaturen angiver er om den nøgle
tilhører vedkommende - ikke om vedkommende er troværdig!


\chapter{Installation af FileZilla}
\label{ex:filezilla}

{\bf Objective:}\\
Installer FileZilla pakken fra \link{http://filezilla-project.org/}

{\bf Suggested method:}\\
Hent installationsprogrammet og installer FileZilla, alternativt WinSCP eller et andet program der forstår SFTP

{\bf Suggested method:}\\

{\bf Solution:}\\

{\bf Discussion:}\\
Det er vigtigt at bruge sikre protokoller når man overfører data, eksempelvis til opdatering af websites







\appendix
\rhead{\fancyplain{}{\bfseries\leftmark}}
%\setlength{\parskip}{5pt}
%\normal

\bibliographystyle{alpha}
%\bibliography{../ipv6-reference/solido-net.bib,../ipv6-reference/rfc.bib,../ipv6-reference/std.bib,../ipv6-reference/fyi.bib}
\bibliography{kramse.bib,rfc.bib,std.bib,fyi.bib}
%,internet.bib}


%\printindex

\end{document}

%%% Local Variables:
%%% mode: latex
%%% TeX-master: t
%%% End:
