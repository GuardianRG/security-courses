\documentclass[Screen16to9,17pt]{foils}
\usepackage{zencurity-slides}
\externaldocument{system-integration-exercises}
\selectlanguage{english}

% Systemintegration

\begin{document}

\mytitlepage
{12. Cloud and Cloud integration}
{KEA System Integration F2020 10 ECTS}



\slide{This weeks Agenda in system integration}

\begin{list2}
\item Follow the plan:\\
\url{https://zencurity.gitbook.io/kea-it-sikkerhed/system-integration/lektionsplan}
\item Plan for May 25.\\
I will go through the last subjects from the last book
\item Talk about exam, online exam, subjects
\end{list2}

\slide{Goals for today}

\hlkimage{6cm}{thomas-galler-hZ3uF1-z2Qc-unsplash.jpg}

Todays goals:
\begin{list2}
\item Finish the Camel book
\item Repeat some stuff, relate it, get an overview og system integration
\item Talk about exam, online exam, subjects
\end{list2}

Photo by Thomas Galler on Unsplash


\slide{Time schedule}
\begin{list2}
\item 08:30 2x 45 min with 10min break\\
Kubernetes\\
Camel chapter 18: Microservices with Docker and Kubernetes
\item 10:15 2x 45 min with 10min break\\
Repeat stuff, relate to other parts\\
Try to tie everything together -- get an overview
\item 12:30 2x 45 min with 10min break \\
Talk about exam, online exam, subjects
\item 14:15 45 min\\
Questions and exercises
\end{list2}





\slide{Plan for today}

\begin{list2}
\item Microservices with Docker and Kubernetes
\item Cloud and Cloud integration
\item Running Camel on Docker
\item Getting started with Kubernetes -- not running Kubernetes today
\end{list2}

Exercises
\begin{list2}
\item Running Java microservices on Docker
\item Getting started with Kubernetes -- run Minikube
\end{list2}




\slide{Part I 08:30 2x 45 min}

Kubernetes

Microservices with Docker and Kubernetes


\slide{Docker }

%\hlkimage{}{}

\begin{quote}
  Package Software into Standardized Units for Development, Shipment and Deployment
  A container is a standard unit of software that packages up code and all its dependencies so the application runs quickly and reliably from one computing environment to another. A Docker container image is a lightweight, standalone, executable package of software that includes everything needed to run an application: code, runtime, system tools, system libraries and settings.
\end{quote}
Source: \\{\footnotesize
\url{https://www.docker.com/resources/what-container}}

\begin{list2}
  \item One of the most popular deployment methods today
\end{list2}

\slide{Containerized Applications}

\hlkimage{8cm}{container-what-is-container.png}

Source: {\footnotesize
\url{https://www.docker.com/resources/what-container}}

\begin{list2}
  \item See also \link{https://en.wikipedia.org/wiki/Linux_namespaces}\\
   \emph{Various container software use Linux namespaces in combination with cgroups to isolate their processes, including Docker[11] and LXC.}
\end{list2}

\slide{Docker Images and layers}

\hlkimage{10cm}{container-layers.jpg}

\begin{quote}
A Docker image is built up from a series of layers. Each layer represents an instruction in the image’s Dockerfile. Each layer except the very last one is read-only.
\end{quote}

Source: {\footnotesize
\link{https://docs.docker.com/storage/storagedriver/}}



\slide{Kubernetes}

%\hlkimage{}{}

\begin{quote}
  Kubernetes (K8s) is an open-source system for automating deployment, scaling, and management of containerized applications.
  It groups containers that make up an application into logical units for easy management and discovery. Kubernetes builds upon 15 years of experience of running production workloads at Google, combined with best-of-breed ideas and practices from the community.
\end{quote}
Source: {\footnotesize
\link{https://kubernetes.io/}}

Key points:
\begin{list2}
\item Open source originally from Google
\item Scalable
\item Uses containers inside
\item Infrastructure as code
\end{list2}


\slide{Infrastructure as code}

%\hlkimage{}{}

\begin{quote}
{\bf Infrastructure as code (IaC)} is the process of managing and provisioning computer data centers through machine-readable definition files, rather than physical hardware configuration or interactive configuration tools.[1] The IT infrastructure managed by this comprises both physical equipment such as bare-metal servers as well as virtual machines and associated configuration resources. The definitions may be in a version control system. It can use either scripts or declarative definitions, rather than manual processes, but the term is more often used to promote declarative approaches.
\end{quote}
Source: {\footnotesize
\link{https://en.wikipedia.org/wiki/Infrastructure_as_code}}

\begin{list2}
  \item Has become the norm in many places
\end{list2}


\slide{Exercise: Lets run Kubernetes}

\hlkimage{10cm}{kubernetes-minikube.png}

\begin{list2}
  \item \link{https://kubernetes.io/docs/tutorials/hello-minikube/}
\end{list2}



\slide{Camel chapter 18: Microservices with Docker and Kubernetes}

This chapter covers
\begin{list2}
\item Running Camel on Docker
\item Getting started with Kubernetes
\item Running and debugging Camel on Kubernetes
\item Understanding essential Kubernetes concepts
\item Building resilient microservices on Kubernetes
\item Testing Camel microservices on Kubernetes
\item Introducing fabric8, Kubernetes Helm, and OpenShift
\end{list2}

Source: {\footnotesize\\
\emph{Camel in action}, Claus Ibsen and Jonathan Anstey, 2018, 2nd edition
ISBN: 978-1-61729-293-4}



\slide{18.1.2 Running Camel on Docker}

%\hlkimage{}{}


  the Dockerfile you’ll use to run the Spring Boot client microservice contains
  just three lines of text:
\begin{alltt}
  FROM openjdk:latest
  COPY maven /maven/
  CMD java -jar maven/spring-­docker-2.0.0.jar
\end{alltt}

  A Docker image is a compressed TAR file that includes the Dockerfile in the root
  alongside other files you want to include in the image. The Spring Boot Docker image
  consists of only two files:
\begin{alltt}
  maven/spring-­docker-2.0.0.jar
  Dockerfile
\end{alltt}

\begin{list2}
  \item We have seen problems with various JDK versions
  \item Running on Docker might be simpler
  \item Help available: {\footnotesize\link{https://docs.docker.com/develop/develop-images/dockerfile_best-practices/}}
\end{list2}



\slide{Getting started with Kubernetes: Minikube}


\begin{alltt}
  minikube start --cpus 2 --memory 2048 --disk-­size 10g
\end{alltt}
The last parameter is important; it specifies which VM driver to use (see Minikube documentation for details). After the installation is complete, you can get the status of Minikube:
\begin{alltt}
  $ minikube status
  minikubeVM: Running
  localkube: Running
  kubectl: Correctly Configured: pointing to minikube-­vm at 192.168.64.2
\end{alltt}

This means the local Kubernetes cluster is up and running.

\begin{list2}
\item We already saw Minikube running in our browser
\end{list2}


\slide{Running and debugging Camel on Kubernetes}

%\hlkimage{}{}

\begin{quote} {\bf
18.3 Running Camel and other applications in Kubernetes}
When you {\bf run applications\\
on Kubernetes, they run as containers loaded from Docker images.} The information you learned in the previous section about running Camel on Docker is required knowledge for working with Kubernetes.
\end{quote}


\slide{Understanding essential Kubernetes concepts}

%\hlkimage{}{}

\begin{alltt}
$ kubectl get deployment -o yaml hello-­world
apiVersion: extensions/v1beta1
kind: Deployment
metadata:
  annotations:
    deployment.kubernetes.io/revision: "1"
  creationTimestamp: 2017-11-04T20:33:36Z
  generation: 1
  labels:
    foo: bar
  name: hello-­world
  ...
\end{alltt}

\begin{list2}
\item Example YAML file from Kubernetes
\end{list2}




\slide{Building resilient microservices on Kubernetes}

\hlkimage{18cm}{kubernetes-scaling-up.png}

\begin{list2}
\item Kubernetes can be told to create more pods/containers
\item AND can check if it alive and good
\end{list2}



\slide{Introducing fabric8, Kubernetes Helm, and OpenShift}

%\hlkimage{}{}

\begin{quote}

\end{quote}

Book lists multiple tools that can help making Java applications Kubernetes-ready:
\begin{list2}
\item Docker Maven plugin
\item Kubernetes-ready fabric8 Maven plugin
\item Kubernetes Helm
\item OpenShift
\end{list2}

We wont go into detail with these, and check if better tools are available before use

\slide{Securing Kubernetes}

%\hlkimage{}{}

\begin{quote}
  Attacking and Defending Kubernetes, with Ian Coldwater\\
  Ian Coldwater specializes in breaking and hardening Kubernetes, containers, and cloud native infrastructure. A pre-eminent voice in the Kubernetes security community, Ian is currently a Lead Platform Security Engineer at Heroku. Ian joins Adam Glick and Craig Box to talk about the offensive and defensive arts.
\end{quote}
{\footnotesize\link{https://www.heroku.com/podcasts/kubernetes-podcast-from-google/attacking-and-defending-kubernetes-with-ian-coldwater}}

\begin{list2}
  \item Securing Kubernetes can be hard work
  \item
  \item follow Ian Coldwater, @IanColdwater \link{https://twitter.com/iancoldwater}
\end{list2}



\slide{Part II 10:15 2x 45 min}

Repeat stuff, relate to other parts\\
Try to tie everything together -- get an overview


\slide{Helm Database}

%\hlkimage{}{}

\begin{quote}

\end{quote}

\begin{list2}
  \item Book uses Helm to deploy a database
  \item Easier than running Postgresql yourself?
  \item Do you want your database inside Kubernetes? why/why not
\end{list2}

\slide{Similar thoughts about load balancing}

%\hlkimage{}{}

\begin{quote}

\end{quote}

\begin{list2}
  \item Do we run everything inside the Kubernetes cluster?
  \item Do we want/need hardware acceleration for things like load balancing and HTTPS/TLS termination
\end{list2}


\slide{Part III 12:30 2x 45min}

Online exam: 29/6, 30/6 2020

Lets talk about it


\slide{Definition}

\begin{quote}
  System integration is defined in engineering as the process of bringing together the component sub-systems into one system (an aggregation of subsystems cooperating so that the system is able to deliver the overarching functionality) and ensuring that the subsystems function together as a system,[1] and in information technology[2] as the process of linking together different computing systems and software applications physically or functionally,[3] to act as a coordinated whole.

  The system integrator integrates discrete systems utilizing a variety of techniques such as computer networking, enterprise application integration, business process management or manual programming.[4]
\end{quote}

Source:\\
\url{https://en.wikipedia.org/wiki/System_integration}



\slide{Intended Learning Outcomes}

\begin{list1}
\item To get acquainted with the challenges of developing business applications
\item To understand the difference between
\begin{list2}
\item tightly coupled and loosely coupled system
\item synchronous and asynchronous integration
\end{list2}
\item To get an overview of existing technologies and solutions in system integration
\item To get programming practice in developing P2P integration using networking
protocols
\end{list1}

\slide{Course Description}

From: STUDIEORDNING

{\bf Knowledge}

The objective is to give the student knowledge of

\begin{list2}
\item business considerations associated with system integration
\item standards and standardization organizations
\item storage, transformation and integration of data resources
\item techniques used in data conversion and migration
\item the service concept and understanding of its connection with service-oriented architecture
\item technologies that can be used to implement a service-oriented architecture
\item integration tools
\end{list2}

\slide{Skills}

{\bf Skills}

The objective is that the students acquire the ability to

\begin{list2}
\item use object-oriented system in service-oriented architecture
\item design a system for easy integration with other systems and using existing services
\item transform or expand a system, so that it can work in a service-oriented architecture
\item apply patterns that support system integration
\item develop supplementary modules for generic systems
\item integrate generic and other systems
\item choose from different methods of integration
\item translate elements of a business strategy into concrete requirements for system integration
\end{list2}

\slide{Proficiencies}

{\bf Proficiencies}

The objective is that the students acquired proficiency in

\begin{list2}
\item choosing from different integration techniques
\item acquiring knowledge about development in standards for integration
\item adapting IT architecture so that future integration of systems is taken into account
\item converting elements in a business strategy to specific requirements for systems integration
\item adapting a system development method, so that it supports system integration
\end{list2}




\slide{Exam}

\begin{list2}
\item Exam
\item Individual: Oral based on curriculum
\item Graded (7 scale)
\item Draw a question. Question covers a topic
\item Discuss the topic, and use practical examples
\item Exam is 40 minutes in total, including pulling the question and grading
%\item Count on being able to present talk for about 10 minutes
\item Prepare material (keywords, examples, exercises, wireshark captures) for different topics so that you can use it to help you at the exam

\end{list2}




\slide{Primary literature}

\hlkrightpic{4cm}{0cm}{old_book_lumen_design_st_01.png}
Primary literature:
\begin{list2}
\item \emph{Enterprise Integration Patterns}, Gregor Hohpe and Bobby Woolf, 2004\\
ISBN: 978-0-321-20068-6 EIP for short
\item \emph{Camel in action}, Claus Ibsen and Jonathan Anstey, 2018\\
ISBN: 978-1-61729-293-4
\item \emph{Service‑Oriented Architecture: Analysis and Design for Services and Microservices},\\ Thomas Erl, 2017
ISBN: 978-0-13-385858-7
\end{list2}
Supporting literature:
\begin{list2}
\item Various internet resources, to be decided
\end{list2}



\slide{Book: Enterprise Integration Patterns}

\hlkimage{6cm}{eip-book.png}

\emph{Enterprise Integration Patterns}, Gregor Hohpe and Bobby Woolf, 2004\\
ISBN: 978-0-321-20068-6 EIP for short

\slide{Companion Web Site}


\begin{quote}
"That's why Bobby Woolf and I documented a pattern language consisting of 65 integration patterns to establish a technology-independent vocabulary and a visual notation to design and document integration solutions. Each pattern not only presents a proven solution to a recurring problem, but also documents common "gotchas" and design considerations.

The patterns are brought to life with examples implemented in messaging technologies, such as JMS, SOAP, MSMQ, .NET, and other EAI Tools. The solutions are relevant for a wide range of integration tools and platforms, such as IBM WebSphere MQ, TIBCO, Vitria, WebMethods (Software AG), or Microsoft BizTalk, messaging systems, such as JMS, WCF, Rabbit MQ, or MSMQ, ESB's such as Apache Camel, Mule, WSO2, Oracle Service Bus, Open ESB, SonicMQ, Fiorano or Fuse ServiceMix."
\end{quote}

Source:\\
\link{https://www.enterpriseintegrationpatterns.com/}

\slide{Book: Camel in Action}

\hlkimage{5cm}{Ibsen-Camel-2ed-HI.png}

\emph{Camel in action}, Claus Ibsen and Jonathan Anstey, 2018\\
ISBN: 978-1-61729-293-4


\slide{Book: Service-Oriented Architecture}

\hlkimage{5cm}{thomas-erl-book.png}
\emph{Service‑Oriented Architecture: Analysis and Design for Services and Microservices},\\ Thomas Erl, 2017
ISBN: 978-0-13-385858-7


\slide{Technologies used in this course}

The following tools and environments are examples that were introduced in this course:

\begin{list2}
\item Enterprise Integration Patterns (EIP) and Service-oriented Architecture (SOA)
\item Programming languages and frameworks Java, Spring, Python
\item Systems for running integration: Tomcat, Jetty, Docker, Camel
\item Networking and network protocols: TCP/IP, HTTP, DNS, FTP, SMTP
\item Formats XML, XSLT, YAML, JSON, WSDL, GRPC, msgpack, protobuf, apache thrift
\item Web technologies and services: REST, API, SOAP
\item Databases: JDBC, Postgresql, ACID
\item Tools like Maven, Linux, APT, Ansible, Nginx, cURL, Git and Github
\item Message queueing systems: Apache ActiveMQ, RabbitMQ and Redis
\item Aggregated example platforms: Elastic stack, Logstash, Elasticsearch, Kibana
\item Cloud and virtualisation Docker, Kubernetes, Azure, AWS, microservices
\end{list2}






\slide{Exam subjects}

\begin{list2}
\item[1] Enterprise Integration Patterns (EIP)
\item[2] Service-oriented Architecture (SOA)
\item[3] Systems for running integration: {\bf Camel}, Tomcat, Jetty, Docker, Kubernetes
\item[4] Networking and network protocols: TCP/IP, HTTP, DNS, FTP, SMTP
\item[5] Integration formats XML, XSLT, YAML, JSON, WSDL, in relation to system-integration
\item[6] Web technologies and services: REST, API, SOAP
\item[7] Databases: JDBC, Postgresql, ACID, persistence, resilience
\item[8] Toolboxes Maven, Linux, APT, Ansible, Nginx, cURL, Git and Github
\item[9] Message queueing systems: JMS, Apache ActiveMQ, RabbitMQ and Redis
\item[10] Aggregated example platforms: Elastic stack,  Elasticsearch, Logstash, Kibana
\end{list2}

Always relate this to system integration, what part do they play in system integration

\slide{1. Enterprise Integration Patterns (EIP)}

\slide{Companion Web Site}

\begin{quote}
"That's why Bobby Woolf and I documented a pattern language consisting of 65 integration patterns to establish a technology-independent vocabulary and a visual notation to design and document integration solutions. Each pattern not only presents a proven solution to a recurring problem, but also documents common "gotchas" and design considerations.

The patterns are brought to life with examples implemented in messaging technologies, such as JMS, SOAP, MSMQ, .NET, and other EAI Tools. The solutions are relevant for a wide range of integration tools and platforms, such as IBM WebSphere MQ, TIBCO, Vitria, WebMethods (Software AG), or Microsoft BizTalk, messaging systems, such as JMS, WCF, Rabbit MQ, or MSMQ, ESB's such as Apache Camel, Mule, WSO2, Oracle Service Bus, Open ESB, SonicMQ, Fiorano or Fuse ServiceMix."
\end{quote}

Source:\\
\link{https://www.enterpriseintegrationpatterns.com/}


\slide{Why Enterprise Integration Patterns?}

\begin{quote}
Enterprise integration is too complex to be solved with a simple 'cookbook' approach. Instead, patterns can provide guidance by documenting the kind of experience that usually lives only in architects' heads: they are accepted solutions to recurring problems within a given context. Patterns are abstract enough to apply to most integration technologies, but specific enough to provide hands-on guidance to designers and architects. Patterns also provide a vocabulary for developers to efficiently describe their solution.

{\bf
Patterns are not 'invented'; they are harvested from repeated use in practice.} If you have built integration solutions, it is likely that you have used some of these patterns, maybe in slight variations and maybe calling them by a different name. The purpose of this site is not to "invent" new approaches, but to present a coherent collection of relevant and proven patterns, which in total form an integration pattern language.
\end{quote}

Source:\\
\link{https://www.enterpriseintegrationpatterns.com/}


\slide{EIP Patterns}

\hlkimage{16cm}{eip-patterns.png}

\slide{Challenges}

\begin{list2}
\item Networks are unreliable. The internet is always broken, somewhere a link is down, a system being booted etc.
\item Networks are slow. Sending data across networks are slowers than making a local call
\item Any two applications are different. Different programming languages, operating systems, and data formats
\item Change is inevitable. Applications change over time
\item Added: everything is linked, everything uses networking
\end{list2}

\slide{Helpful patterns}

\begin{list2}
\item File Transfer(43)
\item Shared database (47)
\item Remote Procedure Invocation (50) - typically using Remote Procecure Call (RPC)
\item Messaging (53) one application publishes a message to a common message channel, other applications read from the channel
\end{list2}

Source: EIP book


\slide{Camel Components}

\begin{quote}
Components are the primary extension point in Camel. Over the years since Camel’s
inception, the list of components has grown. As of version 2.20.1, Camel ships with
more than 280 components, and dozens more are available separately from other com-
munity sites.
\end{quote}
Source: \emph{Camel in action}, Claus Ibsen and Jonathan Anstey, 2018




\slide{2. Service-oriented Architecture (SOA)}


\slide{Book: Service-Oriented Architecture}

\hlkimage{5cm}{thomas-erl-book.png}
\emph{Service‑Oriented Architecture: Analysis and Design for Services and Microservices},\\ Thomas Erl, 2017
ISBN: 978-0-13-385858-7



\slide{Service-oriented architecture (SOA)}

\begin{quote}
Service-oriented architecture (SOA) is a style of software design where services are provided to the other components by application components, through a communication protocol over a network. A SOA service is a discrete unit of functionality that can be accessed remotely and acted upon and updated independently, such as retrieving a credit card statement online. {\bf SOA is also intended to be independent of vendors, products and technologies.[1]}

A service has four properties according to one of many definitions of SOA:[2]
\begin{list2}
\item It logically represents a business activity with a specified outcome.
\item It is self-contained.
\item It is a black box for its consumers, meaning the consumer does not have to be aware of the service's inner workings.
\item It may consist of other underlying services.[3]
\end{list2}
\end{quote}
Source:{\footnotesize\\
\url{https://en.wikipedia.org/wiki/Service-oriented_architecture}}



\slide{The SOA Manifesto}

Through our work we have come to prioritize:
\begin{list2}
\item {\bf Business value} over technical strategy
\item {\bf Strategic goals} over project-specific benefits
\item {\bf Intrinsic interoperability} over custom integration
\item {\bf Shared services} over specific-purpose implementations
\item {\bf Flexibility} over optimization
\item {\bf Evolutionary refinement} over pursuit of initial perfection
\end{list2}

Book references, in Appendix D the \emph{SOA Manifesto}\\
\url{www.soa-manifesto.org.}

I recommend reading the explanation in \emph{The SOA Manifesto Explored}



\slide{Common technologies}

We already mentioned some of the technologies used for this:
\begin{list2}
\item Extensible Markup Language (XML) used in Web service (WS) with Web Services Description Language (WSDL) and Simple Object Access Protocol (SOAP)
\item Allowing us to do Remote Method Invocation (RMI) aka Remote Procedure Call (RPC)
\item Sometimes incorporating XML schema (XSD), Extensible Stylesheet Language Transformations (XSLT) and even producing HyperText Markup Language (HTML) documents or perhaps JavaScript Object Notation (JSON)
\item Too many acronyms? Use Wikipedia!
\item See the patterns and recognize common use-cases
\item Or let Camel and Python convert data \smiley
\end{list2}

\slide{Services Are Collections of Capabilities}

\hlkimage{16cm}{soabook-3-5-service-collection-capabilities.png}
Source: \emph{Service‑Oriented Architecture: Analysis and Design for Services and Microservices}, Thomas Erl, 2017



\slide{Service-Orientation is a Design Paradigm}

\begin{quote}
A design paradigm is an approach to designing solution logic. When building distributed solution logic, design approaches revolve around a software engineering theory known as the “separation of concerns.” In a nutshell, this theory states that a larger problem is more effectively solved when decomposed into a set of smaller problems or concerns. This gives us the option of partitioning solution logic into capabilities, each designed to solve an individual concern. Related capabilities can be grouped into units of solution logic.
\end{quote}

\begin{list2}
\item A \emph{service composition} is a coordinated aggregate of services
\item A \emph{service inventory} is an independently standardized and governed collection of complementary services within a boundary that represents an enterprise or a meaningful segment of an enterprise.
\end{list2}

Source: \emph{Service‑Oriented Architecture: Analysis and Design for Services and Microservices}, Thomas Erl, 2017


\slide{Problems Solved by Service-Orientation}

\hlkimage{16cm}{soabook-3-14-excess-logic.png}

\begin{list2}
\item Redundant functionality is costly in the long run
\item Integration Becomes a Constant Challenge
\end{list2}

Source: \emph{Service‑Oriented Architecture: Analysis and Design for Services and Microservices}, Thomas Erl, 2017

\slide{The Need for Service-Orientation}

The consistent application of the eight design principles we listed earlier results in the widespread proliferation of the corresponding design characteristics:

\begin{list2}
\item increased consistency in how functionality and data is represented
\item reduced dependencies between units of solution logic
\item reduced awareness of underlying solution logic design and implementation details
\item increased opportunities to use a piece of solution logic for multiple purposes
\item increased opportunities to combine units of solution logic into different
confi gurations
\item increased behavioral predictability
\item  increased availability and scalability
\item increased awareness of available solution logic
\end{list2}

Source: \emph{Service‑Oriented Architecture: Analysis and Design for Services and Microservices}, Thomas Erl, 2017


\slide{Reusable Solution Logic}

\hlkimage{15cm}{soabook-3-17-reusable.png}

Source: \emph{Service‑Oriented Architecture: Analysis and Design for Services and Microservices}, Thomas Erl, 2017

\slide{3.5 Four Pillars of Service-Orientation}

The four pillars of service-orientation are
\begin{list2}
\item Teamwork – Cross-project teams and cooperation are required.
\item Education – Team members must communicate and cooperate based on common knowledge and understanding.
\item Discipline – Team members must apply their common knowledge consistently.
\item Balanced Scope – The extent to which the required levels of Teamwork, Education, and Discipline need to be realized is represented by a meaningful yet manageable scope.
\end{list2}
Source: \emph{Service‑Oriented Architecture: Analysis and Design for Services and Microservices}, Thomas Erl, 2017


\slide{3. Systems for running integration}

 {\bf Camel} Tomcat, Jetty, Docker, Kubernetes


 \slide{Book: Camel in Action}

 \hlkimage{5cm}{Ibsen-Camel-2ed-HI.png}

 \emph{Camel in action}, Claus Ibsen and Jonathan Anstey, 2018\\
 ISBN: 978-1-61729-293-4


 \slide{3.1 Data transformation overview}

 \hlkimage{18cm}{camelbook-3-1-transformation.png}

 \begin{list2}
 \item Data format transformation -- The data format of the message body is transformed
 from one form to another. For example, a CSV record is formatted as XML.
 \item Data type transformation -- The data type of the message body is transformed from
 one type to another. For example, java.lang.String is transformed into javax.
 jms.TextMessage .
 \end{list2}

 \slide{Six ways data transformation typically takes place in Camel}

 \begin{list2}
 \item Data transformation using EIPs and Java
 \item Data transformation using components
 \item Data transformation using data formats
 \item Data transformation using templates
 \item Data type transformation using Camel’s
 type-converter mechanism
 \item Message transformation in component adapters
 \end{list2}

 \slide{3.3 Transforming XML}

 \hlkimage{13cm}{camelbook-xslt.png}


 \begin{list2}
 \item Transforming XML with XSLT
 \item Transforming XML with object marshaling
 \end{list2}


 \slide{Camel CSV}


 \begin{minted}[fontsize=\footnotesize]{java}
 from("file://rider/csvfiles")
   .unmarshal().csv()
   .split(body()).to("jms:queue:csv.record");
 \end{minted}

 \begin{list2}
 \item Camel has built-in support for many formats
 \item CSV is a very basic method of moving data
 \end{list2}


 \slide{Chapter 1: file-copy example}

 \begin{alltt}
 hlk@debian-lab:~/projects/system-integration/camelinaction2/chapter1/file-copy$ find data/
 data/
 data/outbox
 data/outbox/message1.xml
 data/inbox
 data/inbox/message1.xml
 \end{alltt}

 \begin{list2}
 \item We want to run the command for Maven to download tools, and \emph{do stuff}
 \item \verb+mvn compile exec:java+
 \item This might take some time!
 \item Note: this is a two step process, so split into \verb+mvn compile+ and \verb+exec:java+ if you have trouble running
 \end{list2}


\slide{Success Execute Java - new files}

\begin{alltt}\footnotesize
$ find data/
data/
data/outbox
data/outbox/message1.xml
data/outbox/message2.txt
data/inbox
data/inbox/message1.xml
data/inbox/message2.txt
\end{alltt}

\begin{list2}
\item We added a new file using editor, and re-ran\\
\verb+echo "some data" > data/inbox/message2.txt+
\end{list2}



 \slide{Apache Tomcat}

 \begin{quote}
 The Apache Tomcat® software is an open source implementation of the Java Servlet, JavaServer Pages, Java Expression Language and Java WebSocket technologies. The Java Servlet, JavaServer Pages, Java Expression Language and Java WebSocket specifications are developed under the Java Community Process.
 \end{quote}

 \begin{list2}
 \item Allows the deployment of web applications J2EE\\ \url{https://en.wikipedia.org/wiki/Java_Platform,_Enterprise_Edition}
 \item Allows the use of Java security policies
 \item Contains the core functionality found in commercial packages
 \item \url{http://Tomcat.apache.org/}
 \end{list2}


 \slide{Java Apps, Tomcat, XML config}

 We will download Apache Tomcat, and perform the following:
 \begin{list2}
 \item Download the software use version 9.0.30\\
 I downloaded \verb+apache-Tomcat-9.0.30.tar.gz+, Windows users take the \verb+.zip+
 \item Unpack and Run the software, see it works
 \item Work with Tomcat manager
 \item Check the configuration - which is in XML
 \item Change the configuration - make the software listen on all IPs, specific IP
 \end{list2}


 \slide{Tomcat Tasks}

 \begin{list2}
 \item Try going to \url{http://127.0.0.1:8080/manager/} - get 401 Unauthorized\\
 Read and fix the problem
 \item Look into the XML file \verb+conf/server.xml+\\
 Change the port 8080 into something else, does port 80/tcp work?
 \item XML configuration files are very common in the Java and system integration space
 \item Tools often found in Java world, Ant, Maven, Tomcat, Spring
 \item Read \url{https://Tomcat.apache.org/Tomcat-9.0-doc/monitoring.html} about Java Management Extensions (JMX) \url{https://en.wikipedia.org/wiki/Java_Management_Extensions}\\
 Note: references to RMI, JMS, SNMP, HTTP etc.
 \end{list2}


\slide{4. Networking and network protocols}

 TCP/IP, HTTP, DNS, FTP, SMTP

After introducing these it would be natural to introduce Camel and how it can support these methods and protocols


\slide{Internet Today}

\hlkimage{10cm}{images/server-client.pdf}

\begin{list1}
\item Clients and servers, roots in the academic world
\item Protocols are old, some more than 20 years
\item Very little is encrypted, mostly HTTPS
\end{list1}


\slide{OSI og Internet modellerne}

\hlkimage{11cm,angle=90}{images/compare-osi-ip.pdf}

\slide{Packets across the wire or wireless}

\hlkimage{20cm}{ethernet-frame-1.pdf}
\begin{list1}
\item Looking at data as a stream the packets are a pattern laid on top
\item Network technology defines the start and end of a frame, example Ethernet
\item From a lower level we receive a packet, example 1500-bytes from Ethernet driver
\item Operating system masks a lot of complexity
\end{list1}


\slide{TCP three way handshake}

\hlkimage{6cm}{images/tcp-three-way.pdf}

\begin{list2}
\item Session setup is used in some protocols
\item Other protocols like HTTP/2 can perform request in the first packet
\end{list2}

\slide{Well-known port numbers}

\hlkimage{6cm}{iana1.jpg}

\begin{list1}
\item IANA maintains a list of magical numbers in TCP/IP
\item Lists of protocl numbers, port numers etc.
\item A few notable examples:
\begin{list2}
\item Port 25/tcp Simple Mail Transfer Protocol (SMTP)
\item Port 53/udp and 53/tcp Domain Name System (DNS)
\item Port 80/tcp Hyper Text Transfer Protocol (HTTP)
\item Port 443/tcp HTTP over TLS/SSL (HTTPS)
\end{list2}
\item Source: \link{http://www.iana.org}
\end{list1}


\slide{Communicate with HTTP}

Try this - use netcat/ncat, available in Nmap package from \url{Nmap.org}:
\begin{alltt}
\small
$ {\bf netcat www.zencurity.com 80
GET / HTTP/1.0}

HTTP/1.1 200 OK
Server: nginx
Date: Sat, 01 Feb 2020 20:30:06 GMT
Content-Type: text/html
Content-Length: 0
Last-Modified: Thu, 04 Jan 2018 15:03:08 GMT
Connection: close
ETag: "5a4e422c-0"
Referrer-Policy: no-referrer
Accept-Ranges: bytes
...
\end{alltt}


\slide{Nginx Load Balancer}


\begin{alltt}\footnotesize
  http \{
      upstream myapp1 \{
          least_conn;
          server srv1.example.com;
          server srv2.example.com;
          server srv3.example.com;
      \}

      server \{
          listen 80;

          location / \{
              proxy_pass http://myapp1;
          \}
      \}
  \}
\end{alltt}

Example from:
\url{http://nginx.org/en/docs/http/load_balancing.html}



\slide{Basic test tools TCP - Telnet and OpenSSL}

\begin{list1}
\item Telnet used for terminal connections over TCP, but is clear-text
\item Telnet can be used for testing connections to many older protocols which uses text commands
\begin{list2}
\item \verb+telnet mail.kramse.dk 25+ create connection to port 25/tcp
\item \verb+telnet www.kramse.dk 80+ create connection to port 80/tcp
\end{list2}
\item Encrypted connections often use TLS and can be tested using OpenSSL command line tool \verb+openssl+
\begin{list2}
\item \verb+openssl s_client -host www.kramse.dk -port 443+\\
create connection to port 443/tcp with TLS
\item \verb+openssl s_client -host mail.kramse.dk -port 993+\\
create connection to port 993/tcp with TLS
\end{list2}
\item Using OpenSSL in client-mode allows the use of the same commands like Telnet after connection
\end{list1}

\slide{Camel SMTP}

Whenever you send an email, you’re using the Simple Mail Transfer Protocol (SMTP)
under the hood. In Camel, an SMTP URI looks like this:
\verb+[smtp|stmps]://[username@]host[:port][?options]+

\hlkimage{16cm}{camelbook-6-8-camel-smtp.png}

SMTP should always be protected with password!

\slide{5. Integration formats XML, XSLT, JSON, WSDL}

Make sure to relate to system-integration

Dont be afraid to talk about your opinion about these


\slide{Data overview XML data, JSON}

\hlkimage{15cm}{chris-lawton-5IHz5WhosQE-unsplash.jpg}

Photo by Chris Lawton on Unsplash

\slide{XML data}

\begin{quote}
  Extensible Markup Language (XML) is a markup language that defines a set of rules for encoding documents in a format that is both human-readable and machine-readable. The World Wide Web Consortium's XML 1.0 Specification[2] of 1998[3] and several other related specifications[4]—all of them free open standards—define XML.[5]

  The design goals of XML emphasize simplicity, generality, and usability across the Internet.[6] It is a textual data format with strong support via Unicode for different human languages. Although the design of XML focuses on documents, the language is widely used for the representation of arbitrary data structures[7] such as those used in web services.
\end{quote}
Source: \url{https://en.wikipedia.org/wiki/XML}

\begin{list2}
\item We have seen XML used for configuration in Apache Tomcat and Camel
\item Perfect for computers, less for humans
\end{list2}

\slide{XML data example - Nmap output}

\begin{minted}[fontsize=\footnotesize]{xml}
  <?xml version="1.0" encoding="UTF-8"?>
  <!DOCTYPE nmaprun>
  <?xml-stylesheet href="file:///usr/bin/../share/nmap/nmap.xsl" type="text/xsl"?>
  <!-- Nmap 7.70 scan initiated Sat Feb 22 23:35:53 2020 as: nmap -oA router -sP 10.0.42.1 -->
  <nmaprun scanner="nmap" args="nmap -oA router -sP 10.0.42.1" start="1582410953"
   startstr="Sat Feb 22 23:35:53 2020" version="7.70" xmloutputversion="1.04">
  <verbose level="0"/>
  <debugging level="0"/>
  <host><status state="up" reason="echo-reply" reason_ttl="62"/>
  <address addr="10.0.42.1" addrtype="ipv4"/>
  <hostnames>
  </hostnames>
  <times srtt="2235" rttvar="5000" to="100000"/>
  </host>
  <runstats><finished time="1582410953" timestr="Sat Feb 22 23:35:53 2020" elapsed="0.32"
   summary="Nmap done at Sat Feb 22 23:35:53 2020; 1 IP address (1 host up)
   scanned in 0.32 seconds" exit="success"/><hosts up="1" down="0" total="1"/>
  </runstats>
  </nmaprun>
\end{minted}


\slide{XML data - documents}

\begin{quote}
Hundreds of document formats using XML syntax have been developed,[8] including RSS, Atom, SOAP, SVG, and XHTML. XML-based formats have become the default for many office-productivity tools, including Microsoft Office (Office Open XML), OpenOffice.org and LibreOffice (OpenDocument), and Apple's iWork[citation needed]. XML has also provided the base language for communication protocols such as XMPP. Applications for the Microsoft .NET Framework use XML files for configuration, and property lists are an implementation of configuration storage built on XML.[9]
\end{quote}
Source: \url{https://en.wikipedia.org/wiki/XML}

\begin{list2}
\item Document formats using XML may still be proprietary!
\item Documents using XML can be validated, are they well-formed according to the Document Type Definition (DTD)
\end{list2}

\slide{XML data - standards}

\begin{quote}
Many industry data standards, such as Health Level 7, OpenTravel Alliance, FpML, MISMO, and National Information Exchange Model are based on XML and the rich features of the XML schema specification. Many of these standards are quite complex and it is not uncommon for a specification to comprise several thousand pages.[citation needed] In publishing, Darwin Information Typing Architecture is an XML industry data standard. XML is used extensively to underpin various publishing formats.
\end{quote}
Source: \url{https://en.wikipedia.org/wiki/XML}

\begin{list2}
\item Dont worry too much about standards at this time
\item The important standards will often be defined by the business area
\end{list2}


\slide{XML data for Service-oriented architecture (SOA)}

\begin{quote}
XML is widely used in a Service-oriented architecture (SOA). Disparate systems communicate with each other by exchanging XML messages. The message exchange format is standardised as an XML schema (XSD). This is also referred to as the canonical schema. XML has come into common use for the interchange of data over the Internet. IETF RFC:3023, now superseded by RFC:7303, gave rules for the construction of Internet Media Types for use when sending XML. It also defines the media types application/xml and text/xml, which say only that the data is in XML, and nothing about its semantics.
\end{quote}
Source: \url{https://en.wikipedia.org/wiki/XML}


\slide{Transforming XML using XSLT}
\begin{quote}


XSLT (Extensible Stylesheet Language Transformations) is a language for transforming XML documents into other XML documents,[1] or other formats such as HTML for web pages, plain text or XSL Formatting Objects, which may subsequently be converted to other formats, such as PDF, PostScript and PNG.[2] XSLT 1.0 is widely supported in modern web browsers.[3]
\end{quote}
Source: \url{https://en.wikipedia.org/wiki/XSLT}

\begin{list2}
\item Can be seen as a mapping between formats, different XML schemas
\item Also is Turing complete, is a programming language
\end{list2}


\slide{XSLT example}

\begin{minted}[fontsize=\footnotesize]{xml}
<?xml version="1.0" encoding="UTF-8"?>
<xsl:stylesheet xmlns:xsl="http://www.w3.org/1999/XSL/Transform" version="1.0">
  <xsl:output method="xml" indent="yes"/>
  <xsl:template match="/persons">
    <root> <xsl:apply-templates select="person"/> </root>
  </xsl:template>
  <xsl:template match="person">
    <name username="{@username}"> <xsl:value-of select="name" /> </name>
  </xsl:template>
</xsl:stylesheet>
\end{minted}

\begin{list2}
\item XSLT uses XPath to identify subsets of the source document tree and perform calculations. XPath also provides a range of functions
\item XSLT functionalities overlap with those of XQuery, which was initially conceived as a query language for large collections of XML documents\\
Source: \url{https://en.wikipedia.org/wiki/XSLT}
\end{list2}

\slide{xsltproc example using Nmap}

\begin{alltt}\footnotesize
$ su -
# apt install nmap xsltproc
# nmap -sP -oA /tmp/router 91.102.91.18
# exit
$ xsltproc /tmp/router.xml > /tmp/router.html
$ firefox /tmp/router.html
\end{alltt}


\begin{list2}
\item We can use the command line tool \verb+xlstproc+ for transforming documents
\item \verb+apt install xsltproc+
\item Its part of the package Libxslt \url{https://en.wikipedia.org/wiki/Libxslt}
\vskip 2cm
\item Try installing the tools Nmap and \verb+xsltproc+ and reproduce the above
\item This is an easy tool to try, and very useful too
\end{list2}





\slide{Data overview JSON}

\begin{quote}
JavaScript Object Notation (JSON, pronounced /ˈdʒeɪsən/; also /ˈdʒeɪˌsɒn/[note 1]) is an open-standard file format or data interchange format that uses {\bf human-readable text} to transmit data objects consisting of attribute–value pairs and array data types (or any other serializable value). It is a very common data format, with a diverse range of applications, such as serving as replacement for XML in AJAX systems.[6]
\end{quote}
Source: \url{https://en.wikipedia.org/wiki/JSON}

\begin{list2}
\item I like JSON much better than XML
\item Many web services can supply data in JSON format
\end{list2}

\slide{JSON example}

\begin{minted}[fontsize=\footnotesize]{json}
{
  "first name": "John",
  "last name": "Smith",
  "age": 25,
  "address": {
    "street address": "21 2nd Street",
    "city": "New York",
    "state": "NY",
    "postal code": "10021"
  },
  "phone numbers": [
    {
      "type": "home",
      "number": "212 555-1234"
    },
  ],
}
\end{minted}

\begin{list2}
\item This is a basic JSON document, new data attribute-value pairs can be added\\
Source: \url{https://en.wikipedia.org/wiki/JSON}
\end{list2}



\slide{6. Web technologies and services: REST, API, SOAP}

Introduce W3C, HTTP, then move quickly to the subjects REST, API, SOAP


\slide{Web Services}

  The term Web service (WS) is either:
  \begin{list2}
  \item  a service offered by an electronic device to another electronic device, communicating with each other via the World Wide Web, or
  \item a server running on a computer device, listening for requests at a particular port over a network, serving web documents (HTML, JSON, XML, images), and creating web applications services, which serve in solving specific domain problems over the Web (WWW, Internet, HTTP)
\end{list2}
Source: \url{https://en.wikipedia.org/wiki/Web_service}

\begin{list2}
\item Today a generic name for services using the internet
\item Web servers such as Apache HTTPD, Nginx etc. provide a service to thew internet allowing access using HTTP
\item Source for some parts on this slide, \url{https://en.wikipedia.org/wiki/Web_service}
\end{list2}




\slide{W3C Web Services}

\begin{quote}
A web service is a software system designed to support interoperable machine-to-machine interaction over a network. It has an interface described in a machine-processable format (specifically WSDL). Other systems interact with the web service in a manner prescribed by its description using SOAP-messages, typically conveyed using HTTP with an XML serialization in conjunction with other web-related standards.
\end{quote}
Source -- W3C, Web Services Glossary[3]


\slide{SOAP - Simple Object Access Protocol}

\begin{quote}
SOAP (abbreviation for Simple Object Access Protocol) is a messaging protocol specification for exchanging structured information in the implementation of web services in computer networks. Its purpose is to provide extensibility, neutrality, verbosity and independence. It uses XML Information Set for its message format, and relies on application layer protocols, most often Hypertext Transfer Protocol (HTTP), although some legacy systems communicate over Simple Mail Transfer Protocol (SMTP), for message negotiation and transmission.
\end{quote}
Source: \url{https://en.wikipedia.org/wiki/SOAP}


Utilizes  UDDI (Universal Description, Discovery, and Integration)

\slide{Web Service Explained }

\begin{quote}
The term "Web service" describes a standardized way of integrating Web-based applications using the XML, SOAP, WSDL and UDDI open standards over an Internet Protocol backbone. XML is the data format used to contain the data and provide metadata around it, SOAP is used to transfer the data, WSDL is used for describing the services available and UDDI lists what services are available.
\end{quote}
Source:\url{https://en.wikipedia.org/wiki/Web_service}



\slide{WSDL - Web Services Description Language}

\begin{quote}
  The Web Services Description Language (WSDL /ˈwɪz dəl/) is an XML-based interface description language that is used for describing the functionality offered by a web service. The acronym is also used for any specific WSDL description of a web service (also referred to as a WSDL file), which provides a machine-readable description of how the service can be called, what parameters it expects, and what data structures it returns. Therefore, its purpose is roughly similar to that of a type signature in a programming language.

  The current version of WSDL is WSDL 2.0. The meaning of the acronym has changed from version 1.1 where the "D" stood for "Definition".
\end{quote}
Source: \url{https://en.wikipedia.org/wiki/Web_Services_Description_Language}

%\begin{list2}
%\item
%\item
%\end{list2}

\slide{WSDL XML}

\begin{minted}[fontsize=\footnotesize]{xml}
<?xml version="1.0" encoding="UTF-8"?>
<description xmlns="http://www.w3.org/ns/wsdl"
             xmlns:tns="http://www.tmsws.com/wsdl20sample"
             xmlns:whttp="http://schemas.xmlsoap.org/wsdl/http/"
             xmlns:wsoap="http://schemas.xmlsoap.org/wsdl/soap/"
             targetNamespace="http://www.tmsws.com/wsdl20sample">

<documentation>
    This is a sample WSDL 2.0 document.
</documentation>
\end{minted}
Source: \url{https://en.wikipedia.org/wiki/Web_Services_Description_Language}



\slide{WSDL XML types}

\begin{minted}[fontsize=\footnotesize]{xml}
<!-- Abstract type -->
   <types>
      <xs:schema xmlns:xs="http://www.w3.org/2001/XMLSchema"
                xmlns="http://www.tmsws.com/wsdl20sample"
                targetNamespace="http://www.example.com/wsdl20sample">

         <xs:element name="request"> ... </xs:element>
         <xs:element name="response"> ... </xs:element>
      </xs:schema>
   </types>

\end{minted}
Source: \url{https://en.wikipedia.org/wiki/Web_Services_Description_Language}

Types	Describes the data. The XML Schema language (also known as XSD) is used (inline or referenced) for this purpose.




\slide{7. Databases: JDBC, Postgresql}

We have used JDBC and Postgresql as examples
What is ACID, persistence, resilience


\slide{File Transfer}

\hlkimage{7cm}{FileTransferIntegration.png}

File Transfer — Have each application produce files of shared data for others to consume, and consume files that others have produced.

Common systems and technologies used:
\begin{list2}
\item File Transfer Protocol (FTP) - old protocol, uses clear text password - should not be used, but still is
\item SFTP/SCP - replaces FTP, Secure FTP/ Secure Copy is part of the Secure Shell (SSH) protocol - available since 1995
\item Hyper Text Transfer Protocol / HTTP Secure (HTTP/HTTPS) - web based protocols
\end{list2}


\slide{Using Persistence}

\hlkimage{13cm}{camelbook-5-5-fail-complete.png}

\slide{Camel database support}

In pretty much every enterprise-­level application, you need to integrate with a database at some point, so it makes sense that Camel has first-­class support for accessing databases. Camel has five components that let you access databases in various ways:
\begin{list2}
\item \emph{JDBC component} -- Allows you to access JDBC APIs from a Camel route.
item SQL component—Allows you to write SQL statements directly into the URI of the
component for using simple queries. This component can also be used for call-
ing stored procedures.
\item \emph{JPA component} -- Persists Java objects to a relational database by using the Java Persistence Architecture.
\item \emph{Hibernate component} -- Persists Java objects by using the Hibernate framework.
This component isn’t distributed with Apache Camel because of licensing
incompatibilities. You can find it at the camel-­extra project (https://github.
com/camel-­extra/camel-­extra).
\item \emph{MyBatis component} -- Allows you to map Java objects to relational databases.
\end{list2}


\slide{Shared Database}

\hlkimage{7cm}{SharedDatabaseIntegration.png}

Shared Database — Have the applications store the data they wish to share in a common database.

Common systems and technologies used:
\begin{list2}
\item database management system (DBMS) using Structured Query Language (SQL), relational database examples:\\
\item PostgresSQL, Oracle DM, Microsoft SQL, MySQL
\url{https://en.wikipedia.org/wiki/SQL}
\item NoSQL databases has been a new input with examples like:
MongoDB, CouchDB, Redis, RIAK\\
\url{https://en.wikipedia.org/wiki/NoSQL}
\end{list2}

\slide{PostgreSQL}

\hlkimage{10cm}{postgresql-short.png}

Relational databases are used around the world for storing production data. When doing system integration projects we will often need to read or store data in databases, so a minimum of knowledge about these are needed.

\begin{list2}
\item  About relational database systems RDBMS \url{https://en.wikipedia.org/wiki/Relational_database}
\item The home page of PostgreSQL \url{https://www.postgresql.org/}
\end{list2}



\slide{ACID}

\begin{quote}
In computer science, {\bf ACID (atomicity, consistency, isolation, durability)} is a set of properties of database transactions intended to guarantee validity even in the event of errors, power failures, etc. In the context of databases, a sequence of database operations that satisfies the ACID properties (and these can be perceived as a single logical operation on the data) is called a transaction. For example, a transfer of funds from one bank account to another, even involving multiple changes such as debiting one account and crediting another, is a single transaction.
\end{quote}

Source: {\footnotesize\\
\link{http://en.wikipedia.org/wiki/ACID}}


\slide{Atomic Transactions}

\begin{quote}
  That’s why the series of events is described as atomic:
  either they all are completed or they all fail—it’s all or nothing. In transactional terms,
  they either \emph{commit} or \emph{roll back}.

\end{quote}
Source: {\footnotesize\\
\emph{Camel in action}, Claus Ibsen and Jonathan Anstey, 2018, 2nd edition
ISBN: 978-1-61729-293-4}

\begin{list2}
\item Books uses Spring TransactionManager
\item I recommend using available - and mature solutions like this -- don't write your own if you can avoid it
\item Which one is up to you though!
\end{list2}


\slide{JMS to Database example}

\hlkimage{20cm}{camelbook-jdb-order.png}

\begin{minted}[fontsize=\footnotesize]{java}
from("jms:accounting")
.to("bean:orderToSql")
.to("jdbc:dataSource?useHeadersAsParameters=true");
\end{minted}

\slide{Uses a bean for mapping}

Listing 6.3   A bean that converts an incoming order to a SQL statement:
\begin{minted}[fontsize=\footnotesize]{java}
public class OrderToSqlBean {
    public String toSql(@XPath("order/@name") String name,
                        @XPath("order/@amount") int amount,
                        @XPath("order/@customer") String customer,
                        @Headers Map<String, Object> outHeaders) {
        outHeaders.put("partName", name);
        outHeaders.put("quantity", amount);
        outHeaders.put("customer", customer);
        return "insert into incoming_orders"
            + "(part_name, quantity, customer) values"
            + " (:?partName, :?quantity, :?customer)";
    }
}
\end{minted}

The last part is an SQL statement doing the \emph{inserting} into the database


\slide{8. Toolboxes }

What are things you need to run and control development within system integration, examples could be:

Maven, Linux, APT, Ansible, Nginx, cURL, Git and Github


\slide{Java tools Needed for Camel Maven}

\begin{quote}
Apache Maven is a software project management and comprehension tool. Based on the concept of a project object model (POM), Maven can manage a project's build, reporting and documentation from a central piece of information.
\end{quote}


\begin{alltt}
hlk@debian-lab:~$ mvn -v
Apache Maven 3.5.4 (1edded0938998edf8bf061f1ceb3cfdeccf443fe; 2018-06-17T20:33:14+02:00)
Maven home: /home/user/projects/system-integration/apache-maven-3.5.4
Java version: 11.0.6, vendor: Debian, runtime: /usr/lib/jvm/java-11-openjdk-amd64
\end{alltt}


\begin{list2}
\item Maven - mvn command
\item  \url{https://en.wikipedia.org/wiki/Apache_Maven}
\end{list2}


\slide{Chapter 1: file-copy example}

\begin{alltt}
hlk@debian-lab:~/projects/system-integration/camelinaction2/chapter1/file-copy$ find data/
data/
data/outbox
data/outbox/message1.xml
data/inbox
data/inbox/message1.xml
\end{alltt}

\begin{list2}
\item We want to run the command for Maven to download tools, and \emph{do stuff}
\item \verb+mvn compile exec:java+
\item This might take some time!
\item Note: this is a two step process, so split into \verb+mvn compile+ and \verb+exec:java+ if you have trouble running
\end{list2}


\slide{Success Execute Java - shortened for slide}

\begin{alltt}\footnotesize
hlk@debian-lab:~/projects/system-integration/camelinaction2/chapter1/file-copy$ mvn exec:java
[INFO] Scanning for projects...
[INFO]
[INFO] ----------------< com.camelinaction:chapter1-file-copy >----------------
[INFO] Building Camel in Action 2 :: Chapter 1 :: File Copy Example 2.0.0
[INFO] --------------------------------[ jar ]---------------------------------
[INFO]
[INFO] --- exec-maven-plugin:1.2.1:java (default-cli) @ chapter1-file-copy ---
[ion.FileCopierWithCamel.main()] DefaultCamelContext  INFO  Apache Camel 2.24.3 (CamelContext: camel-1) is starting
[ion.FileCopierWithCamel.main()] FileEndpoint         INFO  Using default memory based idempotent repository with cache max size: 1000
[ion.FileCopierWithCamel.main()] DefaultCamelContext  INFO  Route: route1 started and consuming from: file://data/inbox?noop=true
[ion.FileCopierWithCamel.main()] DefaultCamelContext  INFO  Total 1 routes, of which 1 are started
[INFO] ------------------------------------------------------------------------
[INFO] BUILD SUCCESS
[INFO] ------------------------------------------------------------------------
[INFO] Total time: 11.908 s
[INFO] Finished at: 2020-02-17T07:11:18+01:00
[INFO] ------------------------------------------------------------------------
\end{alltt}



\slide{Git intro}

\begin{quote}
Git (/ɡɪt/)[7] is a distributed version-control system for tracking changes in source code during software development.[8] It is designed for coordinating work among programmers, but it can be used to track changes in any set of files. Its goals include speed,[9] data integrity,[10] and support for distributed, non-linear workflows.[11]
\end{quote}

Source: \url{https://en.wikipedia.org/wiki/Git}

\begin{list2}
\item We will introduce Git and Github - commercial Git hosting company\\
\url{https://en.wikipedia.org/wiki/Git}
\item Try creating a Git repository, remember to add a README
\item Checkout the repository
\item Edit a file
\item add and commit it
\end{list2}

Use the Github Hello World example: \url{https://guides.github.com/activities/hello-world/}



\slide{9. Message queueing systems}

We have spent a little time talking about
JMS, Apache ActiveMQ, RabbitMQ and Redis



\slide{Messaging}

\hlkimage{7cm}{Messaging.png}

Messaging — Have each application connect to a common messaging system, and exchange data and invoke behavior using messages.

Common systems and technologies used:
\begin{list2}
\item Java Message Service (JMS) API is a Java message-oriented middleware Application Programming Interface (API)
\item Apache ActiveMQ, RabbitMQ, Oracle WebLogic
\item See more at \url{https://en.wikipedia.org/wiki/Message_passing}
\end{list2}



\slide{Enterprise Integration Patterns: Messaging Systems}

\hlkimage{12cm}{camelbook-1-3-1-message.png}

Basic Messaging Concepts
\begin{list2}
\item Channels Messaging applications transmit data through a \emph{Message channel}, a virtual pipe that connects a sender to a receiver.
\item Messages A \emph{message} is an atomic packet of data that can be transmitted on a channel
\item Pipes and filters can perform actions on the messages as they travel through the system, validation or transformed between formats\\
We already saw that last time converting data with Camel
\item Routing a message may have to go through several channels to reach its final destination
\item Transformation, a \emph{message translator} can convert from one format to another
\item Endpoints enable applications to send and receive messages
\end{list2}

\slide{3.1 Data transformation overview}

\hlkimage{18cm}{camelbook-3-1-transformation.png}

\begin{list2}
\item Messages consist of two parts: Header and Body
\item See Internet email, picture formats, packet headers
\item Data is envelopped many times during transmission!
\item Data format transformation -- The data format of the message body is transformed
from one form to another. For example, a CSV record is formatted as XML.
\item Data type transformation -- The data type of the message body is transformed from
one type to another. For example, java.lang.String is transformed into javax.
jms.TextMessage .
\end{list2}


\slide{SOAP Request }

\begin{minted}[fontsize=\footnotesize]{xml}
// HTTP here
POST / HTTP/1.0
Host: localhost:8080
User-agent: SOAPpy xxx (pywebsvcs.sf.net)
Content-type: text/xml; charset=UTF-8
Content-length: 340
SOAPAction: "hello"
// SOAP start here
<?xml version="1.0" encoding="UTF-8"?>
<SOAP-ENV:Envelope
  SOAP-ENV:encodingStyle="http://schemas.xmlsoap.org/soap/encoding/"
  xmlns:SOAP-ENC="http://schemas.xmlsoap.org/soap/encoding/"
  xmlns:SOAP-ENV="http://schemas.xmlsoap.org/soap/envelope/"
>
<SOAP-ENV:Body>
<hello SOAP-ENC:root="1">
</hello>
</SOAP-ENV:Body>
</SOAP-ENV:Envelope>
\end{minted}


\slide{Publish-Subscribe Channel}
\hlkimage{9cm}{EIP-PublishSubscribeSolution.png}

\begin{quote}\footnotesize
A Publish-Subscribe Channel works like this: It has one input channel that splits into multiple output channels, one for each subscriber. When an event is published into the channel, the Publish-Subscribe Channel delivers a copy of the message to each of the output channels. Each output channel has only one subscriber, which is only allowed to consume a message once. In this way, each subscriber only gets the message once and consumed copies disappear from their channels.
\end{quote}

{\footnotesize Source:\\ \url{https://www.enterpriseintegrationpatterns.com/patterns/messaging/PublishSubscribeChannel.html}}


\slide{Message Bus}

\hlkimage{9cm}{EIP-MessageBusSolution.png}

\begin{quote}\footnotesize
A Message Bus is a combination of a common data model, a common command set, and a messaging infrastructure to allow different systems to communicate through a shared set of interfaces. This is analogous to a communications bus in a computer system, which serves as the focal point for communication between the CPU, main memory, and peripherals. Just as in the hardware analogy, there are a number of pieces that come together to form the message bus:
\end{quote}
Source: \url{https://www.enterpriseintegrationpatterns.com/patterns/messaging/MessageBus.html}


\slide{Java Message Service (JMS) Elements}

\begin{list2}
\item JMS provider
An implementation of the JMS interface for message-oriented middleware (MOM). Providers are implemented as either a Java JMS implementation or an adapter to a non-Java MOM.
\item JMS client
An application or process that produces and/or receives messages.
JMS producer/publisher
\item A JMS client that creates and sends messages.
JMS consumer/subscriber
\item A JMS client that receives messages.
JMS message
An object that contains the data being transferred between JMS clients.
\item JMS queue
A staging area that contains messages that have been sent and are waiting to be read (by only one consumer). As the name queue suggests, the messages are delivered in the order sent. A JMS queue guarantees that each message is processed only once.
\item JMS topic
A distribution mechanism for publishing messages that are delivered to multiple subscribers.
\end{list2}
Source: \url{https://en.wikipedia.org/wiki/Java_Message_Service}


\slide{Redis}

\begin{quote}
Redis (/ˈrɛdɪs/;[7][8] Remote Dictionary Server)[7] is an in-memory data structure
project implementing a distributed, in-memory key-value database with optional durability.
\end{quote}

\begin{list2}
\item Often used as a queue, a kind of buffer between systems
\item \url{https://en.wikipedia.org/wiki/Redis}
\item \url{https://redislabs.com}
\end{list2}


\slide{Apache ActiveMQ}

\begin{quote}
Apache ActiveMQ™ is the most popular open source, multi-protocol, Java-based messaging server. It supports industry standard protocols so users get the benefits of client choices across a broad range of languages and platforms. Connectivity from C, C++, Python, .Net, and more is available. Integrate your multi-platform applications using the ubiquitous AMQP protocol. Exchange messages between your web applications using STOMP over websockets. Manage your IoT devices using MQTT. Support your existing JMS infrastructure and beyond. ActiveMQ offers the power and flexibility to support any messaging use-case.
\end{quote}

\url{https://activemq.apache.org/}



\slide{RabbitMQ}

\begin{quote}
RabbitMQ is an open-source message-broker software (sometimes called message-oriented middleware) that originally implemented the Advanced Message Queuing Protocol (AMQP) and has since been extended with a plug-in architecture to support Streaming Text Oriented Messaging Protocol (STOMP), Message Queuing Telemetry Transport (MQTT), and other protocols.[1]

The RabbitMQ server program is written in the Erlang programming language and is built on the Open Telecom Platform framework for clustering and failover. Client libraries to interface with the broker are available for all major programming languages.
\end{quote}

Source: \url{https://en.wikipedia.org/wiki/RabbitMQ}

Home page: \url{https://www.rabbitmq.com/}



\slide{Example from RabbitMQ tutorial: Sending}

Sending\\
The following code fragment establishes a connection, makes sure the recipient queue exists, then sends a message and finally closes the connection.

\inputminted[fontsize=\footnotesize]{python}{programs/sender.py}

Example from: \url{https://www.rabbitmq.com/tutorials/tutorial-one-python.html}

\slide{Example from RabbitMQ tutorial: Receiving}

Receiving\\
Similarly, the following program receives messages from the queue and prints them on the screen:

\inputminted[fontsize=\footnotesize]{python}{programs/recv.py}



\slide{10. Aggregated example platforms: Elastic stack}

What is the benefit of using a large packet with a lot of components, Elasticsearch, Logstash, Kibana


\slide{Elasticsearch}

\begin{quote}
Elasticsearch is a search engine based on the Lucene library. It provides a distributed, multitenant-capable full-text search engine with an HTTP web interface and schema-free JSON documents. Elasticsearch is developed in Java.
\end{quote}

Source: \url{https://en.wikipedia.org/wiki/Elasticsearch}

\begin{list2}
\item Open core means parts of the software are licensed under various open-source licenses (mostly the Apache License)
\item Various browser tools and plugins for ES exist, to make life easier
\item We often use ES for storing Log Messages and Events from multiple systems, a SIEM Security information and event management.
\end{list2}



\slide{Real Example using Elasticsearch}

We will read about a real example:

\begin{quote}
Scaling the Elastic Stack in a Microservices Architecture @ Rightmove\\
Rightmove is the UK's \#1 Property Portal. In the process of helping people find the places they want to live, we serve 55 million requests a day and use Elasticsearch to power our searches and provide our teams with useful analytics to help support our applications.
...
Fast forward to 2017 and we have over 50 microservices all sending their logs to our Elasticsearch cluster. In doing so, we needed a way to scale our configuration on both the hardware and application side of things. So how did we achieve this and what did we learn along the way?
\end{quote}
Source:\\
{\footnotesize\url{https://www.elastic.co/blog/scaling-the-elastic-stack-in-a-microservices-architecture-rightmove}}



\slide{About Elasticsearch}

Look at the example Elasticsearch
\hlkimage{15cm}{elastic-logstash-queue-publish.png}

Source:\\
{\footnotesize\url{https://www.elastic.co/blog/scaling-the-elastic-stack-in-a-microservices-architecture-rightmove}}


\slide{Elastic stack Kibana}

\hlkimage{10cm}{illustrated-screenshot-hero-kibana.png}

Screenshot from \url{https://www.elastic.co/kibana}

Elasticsearch is a search engine and ocument store used in a lot of different systems, allowing cross application integration.

\slide{Getting started with Elastic Stack}

Easy to get started using the tutorial \emph{Getting started with the Elastic Stack} :\\
{\footnotesize\url{https://www.elastic.co/guide/en/elastic-stack-get-started/current/get-started-elastic-stack.html}}

Today Elastic Stack contains lots of different parts.

\begin{list2}
\item Elasticsearch - the core engine
\item Logstash - a tool for parsing logs and other data.\\
\url{https://www.elastic.co/logstash}
\begin{quote}
"Logstash dynamically ingests, transforms, and ships your data regardless of format or complexity. Derive structure from unstructured data with grok, decipher geo coordinates from IP addresses, anonymize or exclude sensitive fields, and ease overall processing."
\end{quote}
\item Kibana - a web application for accessing and working with data in Elasticsearch\\
\url{https://www.elastic.co/kibana}
\end{list2}





\slide{Part IV 14:15 45min}

Questions and exercises







\end{document}
