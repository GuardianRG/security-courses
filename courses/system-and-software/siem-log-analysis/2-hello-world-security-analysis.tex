\documentclass[Screen16to9,17pt]{foils}
\usepackage{zencurity-slides}
\externaldocument{siem-log-analysis-exercises}
\selectlanguage{english}



\begin{document}

\mytitlepage
{2. Hello world of Security Data Analysis}
{KEA Kompetence SIEM and Log Analysis}

\slide{Goals for today}

\hlkimage{6cm}{thomas-galler-hZ3uF1-z2Qc-unsplash.jpg}

Todays goals:
\begin{list2}
\item A Hello World example
\item Try to go from raw data into Zeek, into Elasticsearch
\item Investigating a few data types along the way
\end{list2}

Photo by Thomas Galler on Unsplash




\slide{Plan for today}

\begin{list1}
\item Subjects
\begin{list2}
\item Data types: IP addresses, DNS, Netflow
\item Note: not all details about these types are explained today!
\end{list2}
\item Exercise theme: Low level, high value
\begin{list2}
\item Data types: IP addresses and reputation lists
\item Zeek as an example of a powerful multi-tool
\end{list2}
\end{list1}


\slide{Exercise setup}

We will use a combination of your virtual servers, my switch hardware and my virtual systems.

\vskip 1cm
{\Large \bf There will be sniffing done on traffic!\\
Don't abuse information gathered}

We try to mimic what you would do in your own networks during the exercises.

Another way of running exercises might be:\\
\url{https://github.com/jonschipp/ISLET}

Recommended and used by Zeek and Suricata projects.


\slide{Reading Summary}

\begin{list1}
\item DDS 3. Learning the "Hello World" of Security Data Analysis
\item DDS 4. Performing Exploratory Security Data Analysis \\
Do exercises if you feel like it, notice how small valuable programs can be
\end{list1}


\slide{Reading Summary, continued}

%\hlkimage{}{}

\begin{quote}
This chapter takes the “Hello World” concept and expands it to a walk-through of a self-contained, introductory security data analysis use case that you will be able to follow along with, execute, and take concepts from as you start to perform your own analyses. There are parallel examples in Python and R to provide a somewhat agnostic view of the similarities, strengths, and differences between both languages in a real-life data analysis context. If you’re not familiar with one or both of those languages, you should read Chapter 2 and at least skim some of the external resources referenced there.
\end{quote}
Source: DDS 3. Learning the "Hello World" of Security Data Analysis


\begin{list2}
\item Get data from HTTP -- can be scheduled
\item Process
\item Present it - graphs and later dashboards
\item We dont need to run all examples to see the benefit
\item Lets try some of the scripts from this chapter
\end{list2}


\slide{Reading Summary, continued}

%\hlkimage{}{}

\begin{quote}
This chapter focuses on exploring IP addresses by starting with further analyses on the AlienVault IP Reputation database first seen in Chapter 3.

IP addresses—along with domain names and routing concepts—are the building blocks of the Internet. They are defined in RFC 791, the “Internet Protocol / DARPA Internet Program / Protocol Specification” (http://tools.ietf.org/html/rfc791), which has an elegant and succinct way of describing them:

\hskip 1cm A name \emph{indicates what we seek}. An address \emph{indicates where it is}. A route \emph{indicates how to get there}.

\end{quote}
Source DDS 4. Performing Exploratory Security Data Analysis

{\small
\begin{enumerate}
\item Downloading (if necessary) new data
\item  Parsing/munging and converting the new data into a data frame
\item  Validating the contents and structure of the new data
\item  Extracting or computing relevant information from the new data source
\item  Creating one or more new columns in the existing data frame
\item  Running new analyses
\end{enumerate}
}

\slide{What happens today?}

\begin{list1}
\item Think like a blue team member find hacker traffic
\item Get basic tools running
\item Improve situation
\begin{list2}
\item See where the data end up
\item What kind of data and metadata can we extract
\item How can we collect and make use of it
\item Databases and web interfaces, examples shown
\item Consider what your deployment could be
\end{list2}
\end{list1}

\centerline{Today focus on the lower parts, but user interfaces are important too}

\slide{IP: Internet historically}

\begin{list2}
\item[1961]  L. Kleinrock, MIT packet-switching teori
\item[1962]  J. C. R. Licklider, MIT - notes
\item[1964]  Paul Baran: On Distributed Communications
\item[1969]  ARPANET start 4 nodes
\item[1971]  14 nodes
\item[1973]  Work on Internet Protocols IP started
\item[1973]  Email is about 75\% af ARPANET traffik
\item[1974]  TCP/IP: Cerf/Kahn: A protocol for Packet
        Network Interconnection
\item[1983]  EUUG $\rightarrow$ DKUUG/DIKU connection
\item[1988]  ca. 60.000 systemer på Internet
        The Morris Worm rammer ca. 10\%
\item[2002]  Ialt ca. 130 millioner på Internet
\end{list2}

\slide{Internet historically set -  anno 1969}
\hlkimage{6cm}{1969_4-node_map.png}
%size 2

\begin{list2}
\item Node 1: University of California Los Angeles
\item Node 2: Stanford Research Institute
\item Node 3: University of California Santa Barbara
\item Node 4: University of Utah
%\item Kilde: \link{http://www.zakon.org/robert/internet/timeline/}
\end{list2}



\slide{What are Internet hosts }

\hlkimage{15cm}{potaroo-ipv4-address-report.png}
\centerline{Cumulative RIR address assignments, per RIR}

\begin{list1}
\item Source:
IPv4 Address Report - 28-Jan-2019
\link{http://www.potaroo.net/tools/ipv4/}
\end{list1}


\slide{Packets across the wire or wireless}

\hlkimage{20cm}{ethernet-frame-1.pdf}
\begin{list1}
\item Looking at data as a stream the packets are a pattern laid on top
\item Network technology defines the start and end of a frame, example Ethernet
\item From a lower level we receive a packet, example 1500-bytes from Ethernet driver
\item Operating system masks a lot of complexity
\end{list1}


\slide{IPv4 packets - header - RFC-791}

\begin{alltt}\small
    0                   1                   2                   3
    0 1 2 3 4 5 6 7 8 9 0 1 2 3 4 5 6 7 8 9 0 1 2 3 4 5 6 7 8 9 0 1
   +-+-+-+-+-+-+-+-+-+-+-+-+-+-+-+-+-+-+-+-+-+-+-+-+-+-+-+-+-+-+-+-+
   |Version|  IHL  |Type of Service|          Total Length         |
   +-+-+-+-+-+-+-+-+-+-+-+-+-+-+-+-+-+-+-+-+-+-+-+-+-+-+-+-+-+-+-+-+
   |         Identification        |Flags|      Fragment Offset    |
   +-+-+-+-+-+-+-+-+-+-+-+-+-+-+-+-+-+-+-+-+-+-+-+-+-+-+-+-+-+-+-+-+
   |  Time to Live |    Protocol   |         Header Checksum       |
   +-+-+-+-+-+-+-+-+-+-+-+-+-+-+-+-+-+-+-+-+-+-+-+-+-+-+-+-+-+-+-+-+
   |                       Source Address                          |
   +-+-+-+-+-+-+-+-+-+-+-+-+-+-+-+-+-+-+-+-+-+-+-+-+-+-+-+-+-+-+-+-+
   |                    Destination Address                        |
   +-+-+-+-+-+-+-+-+-+-+-+-+-+-+-+-+-+-+-+-+-+-+-+-+-+-+-+-+-+-+-+-+
   |                    Options                    |    Padding    |
   +-+-+-+-+-+-+-+-+-+-+-+-+-+-+-+-+-+-+-+-+-+-+-+-+-+-+-+-+-+-+-+-+

                    Example Internet Datagram Header
\end{alltt}


\slide{Common Address Space}

\vskip 2 cm
\hlkimage{13cm}{IP-address.pdf}

\begin{list1}
\item Internet is defined by the address space, one
\item Based on 32-bit addresses, example 10.0.0.1
\end{list1}

\slide{IPv4 address}

\begin{alltt}
hlk@bigfoot:hlk$ ipconvert.pl 127.0.0.1
Adressen er: 127.0.0.1
Adressen er: 2130706433
hlk@bigfoot:hlk$ ping 2130706433
PING 2130706433 (127.0.0.1): 56 data bytes
64 bytes from 127.0.0.1: icmp_seq=0 ttl=64 time=0.135 ms
64 bytes from 127.0.0.1: icmp_seq=1 ttl=64 time=0.144 ms
\end{alltt}

\begin{list1}
\item IP-adresser typically written as decimal numbers with dots
\item {\bf dot notation}: 10.1.2.3
\end{list1}

\slide{IP-adresser as bits}

\begin{alltt}
IP-adresse: 127.0.0.1
Heltal:	2130706433
Binary:	1111111000000000000000000000001
\end{alltt}

\begin{list1}
\item IP-address converted to bits
\item Computers use bits
\end{list1}

\slide{Internet ABC}

\begin{list1}
\item Previously we used classes: A, B, C, D og E
\item This proved to be a bit inflexible:
\begin{list2}
\item A-klasse has 16 million hosts
\item B-klasse about 65.000 hosts
\item C-klasse only 250 hosts
\end{list2}
\item Most people asked for B-klasser - starting to run out!
\item D-klasse used for multicast
\item E-klasse reserved
\item See \link{http://en.wikipedia.org/wiki/Classful\_network}
\end{list1}

\vskip 5mm
\centerline{\bf Stop saying C, say /24}

\slide{CIDR Classless Inter-Domain Routing}

\hlkimage{13cm}{CIDR-aggregation.pdf}

\begin{list1}
\item Subnet mask originally inferred by the class
\item Started to allocate multiple C-class networks - save remaining B-class\\
Resulted in routing table explosion
\item A subnet mask today is a row of 1-bit
\item 10.0.0.0/24 means the network 10.0.0.0 with subnet mask 255.255.255.0
\item Supernet, supernetting
\end{list1}


\slide{RFC-1918 Private Networks}

\begin{list1}
\item Der findes et antal adresserum som alle må benytte frit:
\begin{list2}
\item 10.0.0.0    -  10.255.255.255  (10/8 prefix)
\item 172.16.0.0  -  172.31.255.255  (172.16/12 prefix)
\item 192.168.0.0 -  192.168.255.255 (192.168/16 prefix)
\end{list2}
\item Address Allocation for Private Internets RFC-1918 adresserne!
\item NB: man må ikke sende pakker ud på internet med disse som afsender, giver ikke mening
\end{list1}

\begin{alltt}
The blocks 192.0.2.0/24 (TEST-NET-1), 198.51.100.0/24 (TEST-NET-2),
and 203.0.113.0/24 (TEST-NET-3) are provided for use in
documentation.

169.254.0.0/16 has been ear-marked as the IP range to use for end node
auto-configuration when a DHCP server may not be found
\end{alltt}



\slide{UDP User Datagram Protocol}
\hlkimage{14cm}{udp-1.pdf}
\begin{list1}
\item Connection-less RFC-768, \emph{connection-less}
\item Used for Domain Name Service (DNS)
\end{list1}

\slide{TCP Transmission Control Protocol}
\hlkimage{12cm}{tcp-1.pdf}

\begin{list1}
\item Connection oriented RFC-791 September 1981, \emph{connection-oriented}
\item Either data delivered in correct order, no data missing, checksum or an error is reported
\item Used for HTTP and others
\end{list1}

\slide{TCP three way handshake}

\hlkimage{6cm}{images/tcp-three-way.pdf}

\begin{list2}
\item Session setup is used in some protocols
\item Other protocols like HTTP/2 can perform request in the first packet
\end{list2}


\slide{Well-known port numbers}

\hlkimage{6cm}{iana1.jpg}

\begin{list1}
\item IANA maintains a list of magical numbers in TCP/IP
\item Lists of protocl numbers, port numers etc.
\item A few notable examples:
\begin{list2}
\item Port 25/tcp Simple Mail Transfer Protocol (SMTP)
\item Port 53/udp and 53/tcp Domain Name System (DNS)
\item Port 80/tcp Hyper Text Transfer Protocol (HTTP)
\item Port 443/tcp HTTP over TLS/SSL (HTTPS)
\end{list2}
\item Source: \link{http://www.iana.org}
\end{list1}



\slide{Book: Practical Packet Analysis (PPA)}
\hlkimage{6cm}{PracticalPacketAnalysis3E_cover.png}

\emph{Practical Packet Analysis,
Using Wireshark to Solve Real-World Network Problems}
by Chris Sanders, 3rd Edition
April 2017, 368 pp.
ISBN-13:
978-1-59327-802-1

\link{https://nostarch.com/packetanalysis3}


\slide{Investigate examples from the internet}

\hlkimage{9cm}{ripe-ncc-atlas-json.png}

\begin{list2}
\item Which web services do you use? Can we find examples of XML and JSON web services
%\item Can we run SOAP, perhaps the SOAPpy example
\item Look into the services provided by DBC, what languages, formats, services - look at their Github account DBCDK
\item We will use internet data from RIPE NCC Atlas service, if nothing else:\\
{\footnotesize\url{https://ripe-atlas-tools.readthedocs.io/en/latest/}\\
\url{https://atlas.ripe.net/docs/api/v2/manual/pdf/ripe_atlas_api_V2_manual.pdf}}
\item If using the command line to set the API key:\
\verb+ripe-atlas configure --set authorisation.create=e7fd3981-9592-481e-8bcd-f50d17e65de8+

\end{list2}



\slide{DNS root servers}

\hlkimage{20cm}{root-servers.png}

\link{http://root-servers.org/}


\slide{DNS is important}

\hlkimage{12cm}{using-packetq.png}

Another tool that provides a basic SQL-frontend to PCAP-files\\
\url{https://www.dns-oarc.net/tools/packetq}\\
\url{https://github.com/DNS-OARC/PacketQ}

Going back in time and finding systems that visited a specific domain can explain when and where an infection started.

\slide{Other tools and references}

Deciding on which tool to use, Zeek or PacketQ depends on the situation.

Other tools and references:
\begin{list2}
\item Packetbeat \link{https://www.elastic.co/products/beats/packetbeat}
\item \link{http://securityblog.switch.ch/2013/01/22/using-packetq/}
\item \link{http://jpmens.net/2013/05/27/server-agnostic-logging-of-dns-queries-responses/}
\end{list2}


\slide{Indicators of Compromise and Signatures}

\begin{quote}
An IOC is any piece of information that can be used to objectively describe a
network intrusion, expressed in a platform-independent manner. This could include a simple indicator such as the IP address of a command and control (C2) ser
ver or a complex set of behaviors that indicate that a mail server is being used as a malicious SMTP relay.

When an IOC is taken and used in a platform-specific language or format, such as a Snort Rule or a Zeek-formatted file, it becomes part of a signature. A sign
ature can contain one or more IOCs.
\end{quote}

Source: Applied Network Security Monitoring Collection, Detection, and Analysis, 2014 Chris Sanders

\slide{Reading Summary, False Positives}

\begin{list2}
\item True Positive (TP). An alert that has correctly identified a specific activity. If a signature was designed to detect a certain type of malware, and an alert is generated when that malware is launched on a system, this would be a true positive, which is what we strive for with every deployed signature.Indicators of Compromise and Signatures
\item False Positive (FP). An alert has incorrectly identified a specific activity. If a signature was designed to detect a specific type of malware, and an alert is generated for an instance in which that malware was not present, this would be a false positive.
\item True Negative (TN). An alert has correctly not been generated when a specific activity has not occurred. If a signature was designed to detect a certain type of malware, and no alert is generated without that malware being launched, then this is a true negative, which is also desirable. This is difficult, if not impossible, to quantify in terms of NSM detection.
\item False Negative (FN). An alert has incorrectly not been generated when a specific activity has occurred.
\end{list2}

Source: Applied Network Security Monitoring Collection, Detection, and Analysis, 2014 Chris Sanders

\slide{Reputation-Based Detection}

\hlkimage{4cm}{switch-1.pdf}

\begin{list2}
\item The most basic form of intrusion detection is reputation-based detection
\item Similar concept to block lists for SMTP spam relays
\item I often recommend \link{https://github.com/stamparm/maltrail} as a source of lists
\item Other sources are lists like RIPE NCC delegated, which IP prefixes are handed out in different countries\\
\link{https://ftp.ripe.net/pub/stats/ripencc/delegated-ripencc-extended-latest}\\
\verb+ripencc|DK|ipv4|185.129.60.0|1024|20151130|allocated|+
\item Tool often mentioned are Argus and SiLK \link{https://tools.netsa.cert.org/silk/}\\
If we end up having time today, or another day, we should look into this tool chain also!
\item Old and mature tools have been proven to work
\end{list2}


\slide{IP reputation}

\begin{list1}
\item Zeek documentation Intel framework\\
\link{https://docs.zeek.org/en/stable/frameworks/intel.html}\\
\item Suricata reputation support\\
\link{https://suricata.readthedocs.io/en/suricata-4.0.5/reputation/index.html}
\end{list1}


\exercise{ex:data-types-ip-address}

\exercise{siem:ip-reputation}


\slide{Low level, high value}


\begin{list2}
\item I will introduce some of my favourite tools, and they are low-level

\end{list2}


\slide{Netflow}

\begin{list2}
\item Netflow is getting more important, more data share the same links
\item Accounting is important
\item Detecting DoS/DDoS and problems is essential
\item Netflow sampling is vital information - 123Mbit, but what kind of traffic
\item NFSen is an old but free application
\link{http://nfsen.sourceforge.net/}
\item Currently also investigating sFlow - hopefully more fine grained
\item sFlow, short for "sampled flow", is an industry standard for packet export at Layer 2 of the OSI model, \\
\link{https://en.wikipedia.org/wiki/SFlow}
\end{list2}

\centerline{Netflow is often from routers, we dont have any here}

Also look into Elastiflow! \link{https://github.com/robcowart/elastiflow}



\slide{Collect Network Evidence from the network}

\begin{list1}
\item Network Flows
\item Cisco standard NetFlow version 5 defines a flow as a unidirectional sequence of packets that all share the following 7 values:
\begin{list2}
\item Ingress interface (SNMP ifIndex)
\item IP protocol, Source IP address and Destination IP address
\item Source port for UDP or TCP, 0 for other protocols
\item Destination port for UDP or TCP, type and code for ICMP, or 0 for other protocols
\item IP Type of Service
\end{list2}
\item today Netflow version 9 or IPFIX
\end{list1}

Source: \\{\footnotesize
\link{https://en.wikipedia.org/wiki/NetFlow}\\
\link{https://en.wikipedia.org/wiki/IP_Flow_Information_Export}}

\slide{ Netflow NFSen}

\hlkimage{17cm}{nfsen-udp-flood.png}

\centerline{An extra 100k packets per second from this netflow source (source is a router)}

\slide{Netflow processing from the web interface}

\hlkimage{12cm}{images/nfsen-processing-1.png}

\centerline{Bringing the power of the command line forward}


\slide{The Zeek Network Security Monitor}

\hlkimage{13cm}{zeek-overview.png}

The Zeek Network Security Monitor is not a single tool, more of a
powerful network analysis framework (former name Bro)

{\small Source \url{https://www.zeek.org/}}

\slide{Zeek scripts}

\begin{alltt}\small
global dns_A_reply_count=0;
global dns_AAAA_reply_count=0;
...
event dns_A_reply(c: connection, msg: dns_msg, ans: dns_answer, a: addr)
        \{
        ++dns_A_reply_count;
        \}

event dns_AAAA_reply(c: connection, msg: dns_msg, ans: dns_answer, a: addr)
        \{
        ++dns_AAAA_reply_count;
        \}
\end{alltt}

source: dns-fire-count.bro from\\
{\small \link{https://github.com/LiamRandall/bro-scripts/tree/master/fire-scripts}
\url{https://www.bro.org/sphinx-git/script-reference/scripts.html}}


\exercise{ex:zeekweb}

\slide{Get Started with Zeek}

\begin{list1}
\item To run in “base” mode:
 \verb+bro -r traffic.pcap+
\item To run in a “near broctl” mode:
\verb+bro -r traffic.pcap local+
\item To add extra scripts:
\verb+bro -r traffic.pcap myscript.bro+
\end{list1}

\centerline{Note: the project was renamed from Bro to Zeek in Oct 2018}
\slide{Zeek demo: Run}

\begin{alltt}\small
// install zeek first
kunoichi:~ root# broctl
Hint: Run the broctl "deploy" command to get started.

Welcome to BroControl 1.5
Type "help" for help.

[BroControl] > install
creating policy directories ...
installing site policies ...
generating standalone-layout.bro ...
generating local-networks.bro ...
generating broctl-config.bro ...
generating broctl-config.sh ...
...
kunoichi:etc root# grep eth0 node.cfg
interface=eth0
\end{alltt}

\slide{Zeek demo: Run Zeek}

\begin{alltt}\small
// back to Broctl and start it
[BroControl] > start
starting bro
// and then
kunoichi:bro root# pwd
/usr/local/var/spool/bro
kunoichi:bro root# tail -f dns.log
\end{alltt}

More examples at:\\
\url{https://www.bro.org/sphinx/script-reference/log-files.html}


\slide{Suricata IDS/IPS/NSM}

\hlkimage{6cm}{suricata.png}

\begin{quote}
Suricata is a high performance Network IDS, IPS and Network Security Monitoring engine. Open Source and owned by a community run non-profit foundation, the Open Information Security Foundation (OISF). Suricata is developed by the OISF and its supporting vendors.
\end{quote}

 \link{http://suricata-ids.org/}
 \link{http://openinfosecfoundation.org}

\centerline{I often use Suricata and Zeek together}


\slide{Commercial Support}

You can and should use updated rulesets for Suricata.

I Recommend the Emerging Threats ET Pro ruleset, which is about USD 900 per year per sensor. So two sites with Suricata running in both, 2x USD 900


\exercise{ex:zeekweb}

\exercise{ex:zeekdnsbasic}
\exercise{ex:zeektlsbasic}

\slide{Zeek and Elasticsearch, Filebeat to the rescue}

\begin{list2}
\item Lets try to get data from Zeek into Elasticsearch
\item Can Zeek produce something nice
\item Can Elasticsearch parse something nice
\item I think JSON is nice.
\item Lucky us:
06 AUGUST 2020ENGINEERING
\emph{Collecting and analyzing Zeek data with Elastic Security}
by Michael Young\\
\link{https://www.elastic.co/blog/collecting-and-analyzing-zeek-data-with-elastic-security}
\item PS it talks about GeoIP enrichment too!
\end{list2}



\slide{Zeek packages}

%\hlkimage{}{}

\begin{quote}
\verb+zkg install +
\end{quote}

\begin{list2}
\item Zeek can be expanded and a lot of packages exist
\item \link{https://packages.zeek.org/} 153 packages!
\item Lets look at some example scripts like some from CVEs
\end{list2}


\slide{More Exercises?}

We probably ran out of time, but I wanted to show\\
\link{https://github.com/crits/crits/wiki}
which has a large list of CRITs objects


\slide{Security Onion}

\hlkimage{10cm}{security-onion.png}

Security Onion is a Linux distro for IDS (Intrusion Detection) and NSM (Network Security Monitoring).\\
\link{http://securityonion.net}

\centerline{Nice starting point for researching dashboards/network packets}




\slidenext{}


\end{document}
