\documentclass[Screen16to9,17pt]{foils}
\usepackage{zencurity-slides}
\externaldocument{siem-log-analysis-exercises}
\selectlanguage{english}



\begin{document}

\mytitlepage
{3. Storing and Processing data}
{KEA Kompetence SIEM and Log Analysis}


\slide{Goals for today}

\hlkimage{6cm}{thomas-galler-hZ3uF1-z2Qc-unsplash.jpg}

Todays goals:
\begin{list2}
  \item\end{list2}

  Photo by Thomas Galler on unsplash

\slide{Plan for today}

\begin{list1}
\item Subjects
\begin{list2}
\item
\end{list2}
\item Exercise theme: What is normal
\begin{list2}
\item
\end{list2}
\end{list1}

\slide{Reading Summary}

\begin{list1}
\item CIP 4 A Data-Centric Approach to Security Monitoring
\item Skim read: CIP 7 Tools of the Trade, need to know NetFlow, DNS, and HTTP proxy logs in the real-world
\item Skim read: DDS 8. Breaking Up with Your Relational Database
\end{list1}

\slide{Reading Summary, continued}

%\hlkimage{}{}

\begin{quote}

\end{quote}

\begin{list1}
\item CIP 4 A Data-Centric Approach to Security Monitoring
\begin{list2}
\item
\end{list2}
\end{list1}

\slide{Reading Summary, continued}

%\hlkimage{}{}

\begin{quote}

\end{quote}

\begin{list1}
\item Skim read: CIP 7 Tools of the Trade, need to know NetFlow, DNS, and HTTP proxy logs in the real-world
\begin{list2}
\item
\end{list2}
\end{list1}

\slide{Reading Summary, continued}

%\hlkimage{}{}

\begin{quote}

\end{quote}

\begin{list1}
\item Skim read: DDS 8. Breaking Up with Your Relational Database
\begin{list2}
\item
\end{list2}
\end{list1}


\slide{Subjects: }

\hlkimage{8cm}{homer-end-is-near.jpg}

\begin{list1}
\item
\end{list1}





\slide{Security devops}

\begin{list1}
\item We need devops skillz in security
\item automate, security is also big data
\item integrate tools, transfer, sort, search, pattern matching, statistics, ...
\item tools, languages, databases, protocols, data formats
\item Use GitHub! So many libraries and programs that can help, maybe solve  90\% of your problem, and you can glue the rest together
\item Example introductions:
\begin{list2}
\item Seven languages/database/web frameworks in Seven Weeks
\item Elasticsearch the definitive guide
\end{list2}
\end{list1}

\centerline{We are all Devops now, even security people!}

\slide{}

%\hlkimage{}{}

\begin{quote}

\end{quote}

\begin{list2}
  \item R and Python from DDS book
  - plus libraries and Github

\end{list2}



\slide{Chapter 9: Service API and Contract Design with\\
REST Services and Microservices}


\begin{quote}
REST service contracts are typically designed around the primary functions of HTTP methods, which make the documentation and expression of REST service contracts distinctly different from operation-based Web service contracts. Regardless of the differences in notation, the same overarching contract-first approach to designing REST service contracts is paramount when building services for a standardized service inventory.
\end{quote}

\begin{list2}
\item REST entity service contracts are typically dominated by service capabilities that include inherently idempotent and reliable GET, PUT, or DELETE methods
\item This chapter provides service contract design guidance for service candidates modeled as a result of the service-oriented analysis stage covered in Chapter 7.
\end{list2}
Source: {\footnotesize\\
\emph{Service‑Oriented Architecture: Analysis and Design for Services and Microservices}, Thomas Erl, 2017}




\slide{REST Service}


\hlkimage{12cm}{soabook-9-1-REST.png}

\begin{list2}
\item Very typical REST URL/method \verb+GET /invoice/{invoice-id}+
\end{list2}
Source: {\footnotesize\\
\emph{Service‑Oriented Architecture: Analysis and Design for Services and Microservices}, Thomas Erl, 2017}



\slide{}


\begin{quote}
The following is a series of common guidelines and considerations for designing REST service contracts.
\end{quote}


\begin{list2}
\item Uniform Contract Design Considerations
\item Designing and Standardizing Methods
\item Designing and Standardizing HTTP Headers
\item Designing and Standardizing HTTP Response Codes
\item Customizing Response Codes
\item Designing Media Types
\end{list2}
Source: {\footnotesize\\
\emph{Service‑Oriented Architecture: Analysis and Design for Services and Microservices}, Thomas Erl, 2017}


\slide{Designing and Standardizing HTTP Response Codes}

\begin{list2}
\item 100-199 are informational codes used as low-level signaling mechanisms, such as a confirmation of a request to change protocols
\item 200-299 are general success codes used to describe various kinds of success conditions
\item 300-399 are redirection codes used to request that the consumer retry a request to a different resource identifier, or via a different intermediary
\item 400-499 represent consumer-side error codes that indicate that the consumer has produced a request that is invalid for some reason, example 404 file not found
\item 500-599 represent service-side error codes that indicate that the consumer’s request may have been valid but that the service has been unable to process it for internal reasElasticsearch exposes REST APIs that are used by the UI components and can be called directly to configure and access Elasticsearch features.ons.
\end{list2}
Source: {\footnotesize\\
\emph{Service‑Oriented Architecture: Analysis and Design for Services and Microservices}, Thomas Erl, 2017}



\slide{Elasticsearch}

\begin{quote}
Elasticsearch is a search engine based on the Lucene library. It provides a distributed, multitenant-capable full-text search engine with an HTTP web interface and schema-free JSON documents. Elasticsearch is developed in Java.
\end{quote}

Source: \url{https://en.wikipedia.org/wiki/Elasticsearch}

\begin{list2}
\item Open core means parts of the software are licensed under various open-source licenses (mostly the Apache License)
\item Various browser tools and plugins for ES exist, to make life easier
\item I often use ES for storing Log Messages and Events from multiple systems, a SIEM Security information and event management.
\end{list2}


\slide{Elasticsearch SIEM}

%\hlkimage{}{}

\begin{quote}

\end{quote}

\begin{list2}
  \item
\end{list2}


\slide{Elasticsearch REST}

\begin{quote}

\end{quote}

\begin{list2}
\item Elasticsearch exposes REST APIs that are used by the UI components and can be called directly to configure and access Elasticsearch features.
\item \link {https://www.elastic.co/guide/en/elasticsearch/reference/current/rest-apis.html}
\item So REST is used for putting data, using \verb+PUT+ and \verb+POST+
\item And REST is used for getting data with \verb+GET+, but also getting information about the Elasticsearch system itself, cluster health etc.
\item It supports advanced querying through the API and parallel execution of searches across a cluster of nodes
\end{list2}


\exercise{ex:postman-api}

\exercise{ex:es-rest-api}





Exercises
Theme: What is normal

Logstash JSON etc.

\slide{Now we need Elastic stack running!}

Note: Kibana makes it easy to use sample data, feel free to experiment!

\exercise{ex:postman-api}

\exercise{gettingstartedelastic}

\exercise{ex:basicansible}

\exercise{ex:es-rest-api}

NetFlow Nfdump, DNS, web server logs

Wireshark here? Demonstrate DNS and HTTPS SNI again?

Mention Moloch, which is renamed Arkime


CEF formatet
Common Event Format ?




\slidenext{}


\end{document}
