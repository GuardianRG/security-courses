\documentclass[Screen16to9,17pt]{foils}
\usepackage{zencurity-slides}
\externaldocument{system-security-exercises}
\selectlanguage{english}


\begin{document}

\mytitlepage
{KEA Kompetence Diploma in IT-security}
{}

\slide{High level description}

{\Large Diplom i it-sikkerhed, english: Diploma of IT-Security.}

Sorry, danish only
\begin{quote}
{\bf Diplomuddannelsen i IT-Sikkerhed} er en {\bf erhvervsrettet videregående uddannelse} målrettet ansatte i virksomheder og organisationer med en væsentlig anvendelse af it.

Uddannelsen er udviklet i samarbejde med brancheorganisationer og erhvervsliv for at imødekomme et stort og stigende behov for {\bf medarbejdere med specialiseret viden om it-sikkerhed}.

Diplomuddannelsen i IT-Sikkerhed er en {\bf deltidsuddannelse}, der løbende tilpasses udviklingen inden for it- og cybersikkerhed.

Indholdet i de enkelte fagmoduler vil løbende blive opdateret til at møde de it-udfordringer, som virksomhederne står overfor.
\end{quote}
Source: \link{https://kompetence.kea.dk/uddannelser/it/diplom-i-it-sikkerhed}

\slide{Access Requirements}

\begin{quote}
{\bf Adgangskrav}\\
For at blive optaget på Diplomuddannelsen i it-sikkerhed skal du have en af følgende
uddannelser:

\begin{list2}
\item En relevant uddannelse mindst på niveau med en erhvervsakademiuddannelse, fx
datamatiker, it-teknolog eller tilsvarende

\item En relevant akademiuddannelse
\end{list2}

Desuden skal du have mindst 2 års erhvervserfaring efter din adgangsgivende
uddannelse.
Opfylder du ikke de nævnte krav men har andre tilsvarende kompetencer, kan den
enkelte institution dispensere efter en individuel kompetencevurdering. Kontakt
uddannelsesinstitutionen for yderligere herom.
\end{quote}

realkompetencevurdering -- \emph{what you know already}\\
\link{https://kompetence.kea.dk/uddannelser/studievejledning/realkompetencevurdering}

\slide{Overview Diploma in IT-security - 60 ECTS total}

\hlkimage{17cm}{kea-diplom-oversigt.png}


\slide{Course Data, example}

{\Large\bf Course: Computer Systems Security\\
VF 3 Systemsikkerhed (10 ECTS)}

Teaching dates: tuesdays and thursdays 17:00 - 20:30\\
28/01 2020, 30/01 2020, 04/02 2020, 06/02 2020, 11/02 2020, 13/02 2020, 18/02 2020, 20/02 2020, 25/02 2020, 27/02 2020, 03/03 2020, 05/03 2020, 10/03 2020, 12/03 2020

Exam: 31/03 2020

\slide{Deliverables and Exam}

\begin{list2}
\item Exam
\item Individual: Oral based on curriculum
\item Graded (7 scale)
\item Draw a question with no preparation. Question covers a topic
\item Try to discuss the topic, and use practical examples
\item Exam is 30 minutes in total, including pulling the question and grading
\item Count on being able to present talk for about 10 minutes
\item Prepare material (keywords, examples, exercises) for different topics so that you can use it to help you at the exam

\vskip 5mm
\item Deliverables:
\item 2 Mandatory assignments
\item Both mandatory assignments are required in order to be entitled to the exam.
\end{list2}


\slide{Course Description, example System Security}

From: STUDIEORDNING Diplomuddannelse i it-sikkerhed August 2018\\
Indhold: Den studerende kan udføre, udvælge, anvende, og implementere praktiske
tiltag til sikring af firmaets udstyr og har viden og færdigheder der supportere dette.

{\bf Viden}

Den studerende har viden om:
\begin{list2}
\item Generelle governance principper / sikkerhedsprocedurer
\item Væsentlige forensic processer
\item Relevante it-trusler
\item Relevante sikkerhedsprincipper til systemsikkerhed
\item OS roller ift. sikkerhedsovervejelser
\item Sikkerhedsadministration i DBMS.
\end{list2}

\slide{Færdigheder}

{\bf Færdigheder}

Den studerende kan:
\begin{list2}
\item Udnytte modforanstaltninger til sikring af systemer
\item Følge et benchmark til at sikre opsætning af enhederne
\item Implementere systematisk logning og monitering af enheder
\item Analysere logs for incidents og følge et revisionsspor
\item Kan genoprette systemer efter en hændelse.
\end{list2}

\slide{Kompetencer}

{\bf Kompetencer}

Den studerende kan:
\begin{list2}
\item håndtere enheder på command line-niveau
\item håndtere værktøjer til at identificere og fjerne/afbøde forskellige typer af endpoint trusler
\item håndtere udvælgelse, anvendelse og implementering af praktiske mekanismer til at forhindre, detektere og reagere over for specifikke it-sikkerhedsmæssige hændelser
\item håndtere relevante krypteringstiltag
\end{list2}

Final word is the Studieordning which can be downloaded from\\
{\footnotesize \link{https://kompetence.kea.dk/uddannelser/it-digitalt/diplom-i-it-sikkerhed}\\
\link{Studieordning_for_Diplomuddannelsen_i_IT-sikkerhed_Aug_2018.pdf}}



\slide{Course Materials}

\begin{list1}
\item This material is in multiple parts:
\begin{list2}
%\item Introduktionsmateriale med baggrundsinformation
\item Slide shows - presentation - this file
\item Exercises - PDF which is updated along the way
\end{list2}
\item Additional resources from the internet
\item Note: the presentation slides are not a substitute for reading the books, papers and doing exercises, many details are not shown
\item My materials are open source:
\begin{list2}
\item \link{https://github.com/kramse/security-courses}
\item \link{https://zencurity.gitbook.io/kea-it-sikkerhed/}
\end{list2}
\end{list1}





\slide{Primary literature -- System Security}

\hlkrightpic{5cm}{0cm}{old_book_lumen_design_st_01.png}
Primary literature - not all chapters are curriculumr:
\begin{list2}
\item \emph{Computer Security: Art and Science}, 2nd edition 2019! Matt Bishop ISBN: 9780321712332 1440 pages
\item \emph{Defensive Security Handbook: Best Practices for Securing Infrastructure}, Lee Brotherston, Amanda Berlin ISBN: 978-1-491-96038-7 284 pages
\item \emph{Forensics Discovery}, Dan Farmer, Wietse Venema 2004, Addison-Wesley 240 pages. Can be found at http://www.porcupine.org/forensics/forensic-discovery/ but recommend buying it. Referenced below as FD

\end{list2}
Supporting literature:
\begin{list2}
\item \emph{Linux Basics for Hackers Getting Started with Networking, Scripting, and Security in Kali}. OccupyTheWeb, December 2018, 248 pp. ISBN-13: 978-1-59327-855-7 - shortened LBfH
\item \emph{Kali Linux Revealed  Mastering the Penetration Testing Distribution}
Raphael Hertzog, Jim O'Gorman - shortened KLR
\end{list2}

\slide{Primary literature -- Communication and Network Security}

Primary literature are these three books:
\begin{list2}
\item \emph{Applied Network Security Monitoring Collection, Detection, and Analysis}, 2014 Chris Sanders \\
ISBN: 9780124172081 - shortened ANSM
\item \emph{Practical Packet Analysis - Using Wireshark to Solve Real-World Network Problems}, 3rd edition 2017, \\
Chris Sanders ISBN: 9781593278021 - shortened PPA
\item \emph{Linux Basics for Hackers Getting Started with Networking, Scripting, and Security in Kali}. OccupyTheWeb, December 2018, 248 pp. ISBN-13: 978-1-59327-855-7 - shortened LBfH
\end{list2}

\slide{Primary literature -- Software Security}

\hlkrightpic{5cm}{0cm}{old_book_lumen_design_st_01.png}
Primary literature:
\begin{list2}
\item \emph{The Art of Software Security Testing Identifying Software Security Flaws}, named AoST or the Green Book
Chris Wysopal ISBN: 9780321304865, AoST or the Green Book
\item \emph{Web Application Security}, Andrew Hoffman, 2020, ISBN: 9781492053118 called WAS
\item \emph{Hacking, 2nd Edition: The Art of Exploitation}, Jon Erickson, February 2008, ISBN-13: 9781593271442, called just hacking
\end{list2}

\slide{Book: Defensive Security Handbook (DSH)}

\hlkimage{6cm}{defensive-security-handbook.jpg}

\emph{Defensive Security Handbook: Best Practices for Securing Infrastructure}, Lee Brotherston, Amanda Berlin ISBN: 978-1-491-96038-7


\slide{Book: Linux Basics for Hackers (LBfH)}

\hlkimage{6cm}{LinuxBasicsforHackers_cover-front.png}

\emph{Linux Basics for Hackers
Getting Started with Networking, Scripting, and Security in Kali}
by OccupyTheWeb
December 2018, 248 pp.
ISBN-13:
9781593278557

\link{https://nostarch.com/linuxbasicsforhackers}
Not curriculum but explains how to use Linux



\end{document}
