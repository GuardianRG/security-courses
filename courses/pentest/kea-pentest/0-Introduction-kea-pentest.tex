\documentclass[Screen16to9,17pt]{foils}
\usepackage{zencurity-slides}

\externaldocument{communication-and-network-security-exercises}
\selectlanguage{english}

%VF 3 Netværkspenetrationstest (5 ECTS)
% Indhold
%Den studerende lærer om hvordan en penetration test udføres, samt kan indhente oplysninger om de seneste sårbarheder, og kan benytte sig af de relevante værktøjer til dette formål.
%Viden
%Den studerende viden om og forståelse for:
%* Etiske samt kontraktuelle forhold omkring en penetrationstest.
%* Standardiseringorganisationers og myndigheders krav til og om penetrationstesting
%Færdigheder
%Den studerende kan:
%Tage højde for sikkerhedsaspekter ved at:
%* Anvende relevante metoder ved udførsel af en penetrationstest
%* Udarbejde en angrebsplan ud fra indsamlede oplysninger om et mål
%* Finde sårbarheder i et givet system
% * Dokumentere og rapportere fundne sårbarheder
% Kompetencer
%Den studerende kan:
% * Planlægge en penetration test, samt eksekvere den både ved brug af værktøjer og manuelt.


\begin{document}

\mytitlepage
{0. Introduction}
{KEA Kompetence Penetration Testing}

\hlkprofiluk

\slide{Course Data}

{\Large\bf Course: Penetration testing\\
VF 3 Netværkspenetrationstest (5 ECTS)}


Exam date:  24/11 2020

Teaching dates:\\
 22/10 2020, 27/10 2020, 29/10 2020, 3/11 2020, 5/11 2020, 10/11 2020, 12/11 2020


\slide{Kursusmateriale}

\begin{list1}
\item Dette materiale består af flere dele:
\begin{list2}
%\item Introduktionsmateriale med baggrundsinformation
\item Kursusmaterialet - præsentationen til undervisning - dette sæt
\item Øvelseshæfte med øvelser
\end{list2}
\item Hertil kommer diverse ressourcer fra internet, specielt RFC-dokumenter
\item Bemærk: kursusmaterialet er ikke en substitut for andet materiale, der er udeladt mange detaljer som forklares undervejs, eller kan slås op op internet
\end{list1}

\slide{Deliverables and Exam}


\begin{list2}
\item Exam
\item Individual: Oral based on curriculum
\item Graded (7 scale)
\item Draw a question with no preparation. Question covers a topic
\item Try to discuss the topic, and use practical examples
\item Exam is 30 minutes in total, including pulling the question and grading
\item Count on being able to present talk for about 10 minutes
\item Prepare material (keywords, examples, exercises, wireshark captures) for different topics so that you can use it to help you at the exam

\vskip 5mm
\item Deliverables:
\item 1 Mandatory assignments
\item Mandatory assignments are required in order to be entitled to go to the exam.
\end{list2}

\slide{Fronter Platform}

\hlkimage{11cm}{fronter.png}

KEA use fronter for sharing news during the course:\\
\link{https://kea-fronter.itslearning.com/}

I use the following link:\\
{\bf\url{https://zencurity.gitbook.io/kea-it-sikkerhed/netvaerkspenetrationstest/lektionsplan}}


\slide{Course Description}

From: STUDIEORDNING Diplomuddannelse i it-sikkerhed August 2018

Indhold:\\
Den studerende lærer om hvordan en penetration test udføres, samt kan indhente oplysninger om de seneste sårbarheder, og kan benytte sig af de relevante værktøjer til dette formål.

Viden\\
Den studerende viden om og forståelse for:
\begin{list2}
\item Etiske samt kontraktuelle forhold omkring en penetrationstest.
\item Standardiseringorganisationers og myndigheders krav til og om penetrationstesting
\end{list2}

\slide{}
Færdigheder\\
Tage højde for sikkerhedsaspekter ved at:
\begin{list2}
\item Anvende relevante metoder ved udførsel af en penetrationstest
\item Udarbejde en angrebsplan ud fra indsamlede oplysninger om et mål
\item Finde sårbarheder i et givet system
 \item Dokumentere og rapportere fundne sårbarheder
\end{list2}

Kompetencer
Den studerende kan:
\begin{list2}
\item Planlægge en penetration test, samt eksekvere den både ved brug af værktøjer og manuelt
\end{list2}

% My translation:\\


Final word is the Studieordning which can be downloaded from\\
{\footnotesize \link{https://kompetence.kea.dk/uddannelser/it-digitalt/diplom-i-it-sikkerhed}\\
\link{Studieordning_for_Diplomuddannelsen_i_IT-sikkerhed_Aug_2018.pdf}}



\slide{Expectations alignment}

In groups of 2 students, brainstorm for 5 minutes on what topics you would like to have in this course

Use 5 minutes more on Agreeing on 5 topics and prioritize these 5 topics


\slide{Forudsætninger}

\begin{list1}
\item Dette er en workshop og fuldt udbytte kræver at
  deltagerne udfører praktiske øvelser
\item Kurset anvender Kali Linux til øvelser, men UNIX kendskab
er ikke nødvendigt
\item Øvelserne foregår via virtuel maskine
\begin{list2}
\item Til penetrationstest og det meste Internet-sikkerhedsarbejde er der
følgende forudsætninger
\item Netværkserfaring
\item TCP/IP principper - ofte i detaljer
\item Programmmeringserfaring er en fordel
\item UNIX kendskab er ofte en {\bfseries nødvendighed}\\
- fordi de nyeste værktøjer er skrevet til UNIX i form af Linux og BSD
\vskip 3 mm
\end{list2}
\end{list1}

\slide{Kursusfaciliteter}
.
\hlkrightpic{85mm}{-1cm}{sample-network.png}

\begin{list1}
\item KEA har et trådløst netværk, som kan bruges til de fleste ting
\item Til visse opgaver medbringer jeg et kursusnetværk med:
\begin{list2}
\item Router
\item Switch
\item Wi-Fi Access Point
\end{list2}
\end{list1}

\slide{Primary literature}

Primary literature are these books:
\begin{list2}
\item \emph{Gray Hat Hacking: The Ethical Hacker's Handbook}, fifth edition
Allen Harper and others
ISBN: 978-1-260-10841-5, May 2018, 640 pp.- shortened grayhat
\item \emph{Linux Basics for Hackers Getting Started with Networking, Scripting, and Security in Kali}. OccupyTheWeb, December 2018, 248 pp. ISBN-13: 978-1-59327-855-7 - shortened LBfH
\end{list2}


\slide{Book: Gray Hat Hacking  (Grayhat)}

\hlkimage{4cm}{9781260108415_6-gray-hat.jpg}

\emph{Gray Hat Hacking: The Ethical Hacker's Handbook}, fifth edition

Authors: Allen Harper, Daniel Regalado, Ryan Linn, Stephen Sims, Branko Spasojevic, Linda Martinez, Michael Baucom, Chris Eagle, Shon Harris\\
Published: May 18th 2018
Paperback ISBN: 978-1-260-10841-5 640 pp.

{\footnotesize\link{https://www.mhprofessional.com/9781260108415-usa-gray-hat-hacking-the-ethical-hackers-handbook-fifth-edition-group}}

\slide{Book: Linux Basics for Hackers (LBhF)}

\hlkimage{6cm}{LinuxBasicsforHackers_cover-front.png}

\emph{Linux Basics for Hackers
Getting Started with Networking, Scripting, and Security in Kali}
by OccupyTheWeb
December 2018, 248 pp.
ISBN-13:
9781593278557

\link{https://nostarch.com/linuxbasicsforhackers}


\slide{Book: Kali Linux Revealed (KLR)}

\hlkimage{6cm}{kali-linux-revealed.jpg}

\emph{Kali Linux Revealed  Mastering the Penetration Testing Distribution}

\link{https://www.kali.org/download-kali-linux-revealed-book/}\\
Not curriculum but explains how to install Kali Linux

\exercise{ex:downloadKLR}


%%% Break?

\slide{Hackerlab Setup}

\hlkimage{7cm}{hacklab-1.png}

\begin{list2}
\item Hardware: modern laptop CPU with virtualisation\\
Dont forget to enable hardware virtualisation in the BIOS
\item Software Host OS: Windows, Mac, Linux
\item Virtualisation software: VMware, Virtual box, HyperV pick your poison
\item Hackersoftware: Kali Virtual Machine \link{https://www.kali.org/}
\item Soft targets: Metasploitable, Windows 2000, Windows XP, ...
\end{list2}

\centerline{Having a Debian 9 Stretch will also be recommended, one pr team}

\slide{Wifi Hardware}

Since we are going to be doing exercises, sniffing data it \\
will be an advantage to have a wireless USB network card.
\begin{list2}
\item The following are two recommended models:
\item TP-link TL-WN722N hardware version 2.0 cheap but only support 2.4GHz
\item Alfa AWUS036ACH 2.4GHz + 5GHz Dual-Band and high performing
\item   Both work great in Kali Linux for our purposes.
\end{list2}

I have some available for teams if you dont buy them.


\slide{Aftale om test af netværk}

\vskip 1cm
{\bfseries Straffelovens paragraf 263 Stk. 2. Med bøde eller fængsel
  indtil 6 måneder
straffes den, som uberettiget skaffer sig adgang til en andens
oplysninger eller programmer, der er bestemt til at bruges i et anlæg
til elektronisk databehandling.}

Hacking kan betyde:
\begin{list2}
\item At man skal betale erstatning til personer eller virksomheder
\item At man får konfiskeret sit udstyr af politiet
\item At man, hvis man er over 15 år og bliver dømt for hacking, kan
  få en bøde - eller fængselsstraf i alvorlige tilfælde
\item At man, hvis man er over 15 år og bliver dømt for hacking, får
en plettet straffeattest. Det kan give problemer, hvis man skal finde
et job eller hvis man skal rejse til visse lande, fx USA og
Australien
\item Frit efter: \link{http://www.stophacking.dk} lavet af Det
  Kriminalpræventive Råd
\item Frygten for terror har forstærket ovenstående - så lad være!
\end{list2}



\exercise{ex:basicVM}


\slide{Lab setup and Nmap Workshop}

\begin{list2}
\item We will now do two things:
\item Prepare/finish your lab setup\\
\url{https://github.com/kramse/kramse-labs}

\item Switch to the materials found in my Nmap Workshop and perform some Nmap scans\\
\url{https://github.com/kramse/security-courses/tree/master/courses/pentest/nmap-workshop}
\end{list2}


\slide{Manualsystemet}

\hlkimage{7cm}{images/unix-command-1.pdf}

\begin{quote}
 It is a book about a Spanish guy called Manual. You should read it.
       -- Dilbert
\end{quote}

\begin{list1}
\item Manualsystemet i UNIX er utroligt stærkt!
\item Det SKAL altid installeres sammen med værktøjerne!
\item Det er næsten identisk på diverse UNIX varianter!
\item \verb+man -k+ søger efter keyword, se også \verb+apropos+
\end{list1}

Prøv \verb+man crontab+ og \verb+man 5 crontab+



\slide{En manualside}

\begin{alltt}\footnotesize
\small
NAME
     cal - displays a calendar
SYNOPSIS
     cal [-jy] [[month]  year]
DESCRIPTION
   cal displays a simple calendar.  If arguments are not specified, the cur-
   rent month is displayed.  The options are as follows:
   -j      Display julian dates (days one-based, numbered from January 1).
   -y      Display a calendar for the current year.

The Gregorian Reformation is assumed to have occurred in 1752 on the 3rd
of September.  By this time, most countries had recognized the reforma-
tion (although a few did not recognize it until the early 1900's.)  Ten
days following that date were eliminated by the reformation, so the cal-
endar for that month is a bit unusual.
\end{alltt}

\slide{Kommandolinien på UNIX}

\begin{list1}
\item Shells kommandofortolkere:
  \begin{list2}
    \item sh - Bourne Shell
\item bash - Bourne Again Shell, ofte default på Linux
\item ksh - Korn shell, lavet af David Korn
\item csh - C shell, syntaks der minder om C sproget
\item flere andre, zsh, tcsh
  \end{list2}
\item Svarer til command.com og cmd.exe på Windows
\item Kan bruges som komplette programmeringssprog
\end{list1}

\slide{Kommandoprompten}


\begin{alltt}
\small
[hlk@fischer hlk]$ id
uid=6000(hlk) gid=20(staff) groups=20(staff),
0(wheel), 80(admin), 160(cvs)
[hlk@fischer hlk]$

[root@fischer hlk]# id
uid=0(root) gid=0(wheel) groups=0(wheel), 1(daemon),
2(kmem), 3(sys), 4(tty), 5(operator), 20(staff),
31(guest), 80(admin)
[root@fischer hlk]#
\end{alltt}

\begin{list1}
\item typisk viser et dollartegn at man er logget ind som almindelig bruger
\item mens en havelåge at man er root - superbruger
\end{list1}

\slide{Kommandoliniens opbygning}


\begin{alltt}
echo [-n] [string ...]
\end{alltt}

\begin{list1}
\item Kommandoerne der skrives på kommandolinien skrives sådan:
\begin{list2}
\item Starter altid med kommandoen, man kan ikke skrive \verb+henrik echo+
\item Options skrives typisk med bindestreg foran, eksempelvis \verb+-n+
\item Flere options kan sættes sammen, \verb+tar -cvf+ eller \verb+tar cvf+
\item I manualsystemet kan man se valgfrie options i firkantede
  klammer \verb+[]+
\item Argumenterne til kommandoen skrives typisk til sidst (eller der
  bruges redirection)
\end{list2}
\end{list1}


\slide{Adgang til UNIX}

\begin{center}
\includegraphics[width=4cm]{images/kde.png}
\includegraphics[width=4cm]{images/gnome-logo-large.png}
\end{center}

\begin{list1}
%\item Systemer der minder om UNIX kan idag nemt skaffes
\item Adgang til UNIX kan ske via grafiske brugergrænseflader, eksempelvis
\begin{list2}
\item KDE \link{http://www.kde.org}
\item GNOME \link{http://www.gnome.org}
\item Xfce \link{https://xfce.org/}
  \end{list2}
\item eller kommandolinien
\item Jeg anbefaler XFCE, som også er default på Kali pt. og findes til Debian
\end{list1}


\slidenext{Buy the books!}



\end{document}
