\documentclass[Screen16to9,17pt]{foils}
\usepackage{zencurity-slides}

\externaldocument{communication-and-network-security-exercises}
\selectlanguage{english}

\begin{document}

\mytitlepage
{0. Introduction}
{Communication and Network Security \the\year}

\hlkprofiluk

\slide{Plan for today}

\begin{list2}
\item Create a good starting point for learning
\item Introduce lecturer and students
\item Expectations for this course
\item Literature list walkthrough
\item Prepare tools for the exercises
\item Kali and Debian Linux introduction
\end{list2}

Exercises
\begin{list2}
\item Kali Linux installation
\item Debian Linux installation
\end{list2}
Linux is a toolbox we will use and participants will use virtual machines


\slide{Course Materials}

\begin{list1}
\item This material is in multiple parts:
\begin{list2}
%\item Introduktionsmateriale med baggrundsinformation
\item Slide shows - presentation - this file
\item Exercises - PDF which is updated along the way
\end{list2}
\item Additional resources from the internet
\item Note: the presentation slides are not a substitute for reading the books, papers and doing exercises, many details are not shown
\end{list1}


\slide{Fronter Platform}

\hlkimage{9cm}{fronter.png}

We will use fronter a lot, both for sharing educational materials and news during the course.

You will also be asked to turn in deliverables through fronter

\link{https://kea-fronter.itslearning.com/}

\vskip 5mm
\centerline{If you haven't received login yet, let us know}


\slide{Course Data}

{\Large\bf Course: Communication and Network Security\\
Ob 1 Netværks- og kommunikationssikkerhed (10 ECTS)}

Exam:
Date  June 11. 2020

Teaching dates: 14/4 2020, 16/4 2020, 21/4 2020, 23/4 2020, 28/4 2020, 30/4 2020, 5/5 2020, 7/5 2020, 12/5 2020, 14/5 2020, 19/5 2020, 20/5 2020, 26/5 2020, 28/5 2020

\slide{Deliverables and Exam}


\begin{list2}
\item Exam
\item Individual: Oral based on curriculum
\item Graded (7 scale)
\item Draw a question with no preparation. Question covers a topic
\item Try to discuss the topic, and use practical examples
\item Exam is 30 minutes in total, including pulling the question and grading
\item Count on being able to present talk for about 10 minutes
\item Prepare material (keywords, examples, exercises, wireshark captures) for different topics so that you can use it to help you at the exam

\vskip 5mm
\item Deliverables:
\item 2 Mandatory assignments
\item Both mandatory assignments are required in order to be entitled to the exam.
\end{list2}


\slide{Course Description}

From: STUDIEORDNING Diplomuddannelse i it-sikkerhed August 2018

Indhold:\\
Modulet går ud på at forstå og håndtere netværkssikkerhedstrusler samt implementere og
konfigurere udstyr til samme.

Modulet omhandler forskellig sikkerhedsudstyr (IDS) til monitorering. Derudover vurdering af sikkerheden i et netværk, udarbejdelse af plan til at lukke eventuelle sårbarheder i netværket samt gennemgang af forskellige VPN teknologier.

My translation:\\
The module is centered around network threats and implementing and configuring equipment in this area.

Module includes different security equipment like IDS for monitoring.
The evaluation of security in a network, developing plans for closing security vulnerabilities in the network and a review of various VPN technologies.

Final word is the Studieordning which can be downloaded from\\
{\footnotesize \link{https://kompetence.kea.dk/uddannelser/it-digitalt/diplom-i-it-sikkerhed}\\
\link{Studieordning_for_Diplomuddannelsen_i_IT-sikkerhed_Aug_2018.pdf}}



\slide{Expectations alignment}

My overall goal

\begin{list2}
\item Introduce networking and related security issues
\item Introduce resources, programs, people, authors, documents, sites\\
 that further your exploration into network security
\end{list2}

Please brainstorm for 5 minutes on what topics you would like to have in this course, and write them down on a piece of paper or note program.

Decide on 5 topics and prioritize these 5 topics

and we will have a common presentation and write them down.

\slide{Prerequisites}

\begin{list1}
\item This course includes exercises and getting the most of the course requires the participants to carry out these practical exercises
\item We will use Kali Linux for the exercises but previous Linux and Unix knowledge is not needed
\item It is recommended to use virtual machines for the exercises
\item Network security and most internet related security work has the following requirements:
\begin{list2}
\item Network experience
\item TCP/IP principles - often in more detail than a common user
\item Programming is an advantage, for automating things
\item Some Linux and Unix knowledge is in my opinion a {\bf necessary skill}\\
-- too many new tools to ignore, and lots found at sites like Github and Open Source written for Linux
\end{list2}
\end{list1}

\slide{Course Network}
.
\hlkrightpic{85mm}{-1cm}{sample-network.png}

\begin{list1}
\item I have a course network with me when needed, \\
which has the following systems:
\begin{list2}
\item OpenBSD router
\item Switches Juniper EX2200-C and small TP-Link
\item UniFi AP wireless access-point
\end{list2}
\end{list1}

This will be at my home, and due to remote teaching - we will investigate your networks and scan across the internet to \emph{my servers}!

\slide{Primary literature}

Primary literature are these three books:
\begin{list2}
\item \emph{Applied Network Security Monitoring Collection, Detection, and Analysis}, 2014 Chris Sanders \\
ISBN: 9780124172081 - shortened ANSM
\item \emph{Practical Packet Analysis - Using Wireshark to Solve Real-World Network Problems}, 3rd edition 2017, \\
Chris Sanders ISBN: 9781593278021 - shortened PPA
\item \emph{Linux Basics for Hackers Getting Started with Networking, Scripting, and Security in Kali}. OccupyTheWeb, December 2018, 248 pp. ISBN-13: 978-1-59327-855-7 - shortened LBfH
\end{list2}

Price check around January 2019 - all three can be bought in hardcopy for 1.000-1.100DKK

\slide{Book: Applied Network Security Monitoring (ANSM)}

\hlkimage{5cm}{ansm-book.png}

\emph{Applied Network Security Monitoring: Collection, Detection, and Analysis}
1st Edition

Chris Sanders, Jason Smith
eBook ISBN: 9780124172166
Paperback ISBN: 9780124172081 496 pp.
Imprint: Syngress, December 2013

{\footnotesize\link{https://www.elsevier.com/books/applied-network-security-monitoring/unknown/978-0-12-417208-1}}

\slide{Book: Practical Packet Analysis (PPA)}
\hlkimage{6cm}{PracticalPacketAnalysis3E_cover.png}

\emph{Practical Packet Analysis,
Using Wireshark to Solve Real-World Network Problems}
by Chris Sanders, 3rd Edition
April 2017, 368 pp.
ISBN-13:
978-1-59327-802-1

\link{https://nostarch.com/packetanalysis3}

\slide{Book: Linux Basics for Hackers (LBhf)}

\hlkimage{6cm}{LinuxBasicsforHackers_cover-front.png}

\emph{Linux Basics for Hackers
Getting Started with Networking, Scripting, and Security in Kali}
by OccupyTheWeb
December 2018, 248 pp.
ISBN-13:
9781593278557

\link{https://nostarch.com/linuxbasicsforhackers}


\slide{Book: Kali Linux Revealed (KLR)}

\hlkimage{6cm}{kali-linux-revealed.jpg}

\emph{Kali Linux Revealed  Mastering the Penetration Testing Distribution}

\link{https://www.kali.org/download-kali-linux-revealed-book/}\\
Not curriculum but explains how to install Kali Linux

\exercise{ex:downloadKLR}



%%% Break?

\slide{Hackerlab Setup}

\hlkimage{7cm}{hacklab-1.png}

\begin{list2}
\item Hardware: modern laptop CPU with virtualisation\\
Dont forget to enable hardware virtualisation in the BIOS
\item Software Host OS: Windows, Mac, Linux
\item Virtualisation software: VMware, Virtual box, HyperV pick your poison
\item Hackersoftware: Kali Virtual Machine \link{https://www.kali.org/}
\item Soft targets: Metasploitable, Windows 2000, Windows XP, ...
\end{list2}

\centerline{Having a Debian VM will also be recommended, one pr team}

\slide{Wifi Hardware}

Since we are going to be doing exercises, sniffing data it \\
will be an advantage to have a wireless USB network card.
\begin{list2}
\item The following are two recommended models:
\item TP-link TL-WN722N hardware version 2.0 cheap but only support 2.4GHz
\item Alfa AWUS036ACH 2.4GHz + 5GHz Dual-Band and high performing
\item   Both work great in Kali Linux for our purposes.
\end{list2}

I have some available for teams if you dont buy them.


\slide{Aftale om test af netværk}

\vskip 1cm
{\bfseries Straffelovens paragraf 263 Stk. 2. Med bøde eller fængsel
  indtil 6 måneder
straffes den, som uberettiget skaffer sig adgang til en andens
oplysninger eller programmer, der er bestemt til at bruges i et anlæg
til elektronisk databehandling.}

Hacking kan betyde:
\begin{list2}
\item At man skal betale erstatning til personer eller virksomheder
\item At man får konfiskeret sit udstyr af politiet
\item At man, hvis man er over 15 år og bliver dømt for hacking, kan
  få en bøde - eller fængselsstraf i alvorlige tilfælde
\item At man, hvis man er over 15 år og bliver dømt for hacking, får
en plettet straffeattest. Det kan give problemer, hvis man skal finde
et job eller hvis man skal rejse til visse lande, fx USA og
Australien
\item Frit efter: \link{http://www.stophacking.dk} lavet af Det
  Kriminalpræventive Råd
\item Frygten for terror har forstærket ovenstående - så lad være!
\end{list2}



\exercise{ex:basicVM}

\exercise{ex:basicDebianVM}


\slide{Access Unix}

\begin{center}
\includegraphics[width=4cm]{images/kde.png}
\includegraphics[width=4cm]{images/gnome-logo-large.png}
\end{center}

\begin{list1}
%\item Systemer der minder om UNIX kan idag nemt skaffes
\item Access to Unix and Linux today typically uses a Windowing system, window manager and has evolved into large environments with base applications:
\begin{list2}
\item KDE \link{http://www.kde.org}
\item GNOME \link{http://www.gnome.org}
\item Xfce \link{https://xfce.org/}
  \end{list2}
  \item or the command line, terminals
\end{list1}


\slide{Running commands on the command line}

\begin{alltt}
echo [-n] [string ...]
\end{alltt}

\begin{list1}
\item Commands writting on the command line are written like this:
\begin{list2}
\item Commands are first, the order matters you cannot reverse them like:  \verb+henrik echo+
\item Options most often written with a dash, like \verb+-n+
\item Multiple options can often be collapsed, \verb+tar -cvf+ eller \verb+tar cvf+\\
Just notice some uses an argument too. The example \verb+-f+ needs a filename!
\item In the manual system optional arguments are in brackets \verb+[]+
\item The arguments for the program then comes after the options and typically also are ordered
\end{list2}
\end{list1}


\slide{Unix Command Line Shells}


  \begin{list2}
    \item sh - Bourne Shell
\item bash - Bourne Again Shell, often the default in Linux
\item ksh - Korn shell, originally by David Korn, popular version \verb+pdksh+ public domain ksh
\item csh - C shell, syntax close to the C programming language
\item multiple others exist: zsh, tcsh
  \end{list2}
\begin{list1}
\item Comparable to command.com, cmd.exe and powershell in Windows
\item Also commonly used for small programs, scripts
\item When writing scripts use the characters number sign and exclamation mark (\verb+#!+) in the beginning
\end{list1}

See more in \url{https://en.wikipedia.org/wiki/Shell_(computing)}\\
\url{https://en.wikipedia.org/wiki/Shebang_(Unix)}

\slide{Command prompts}

\begin{alltt}
\small
[hlk@fischer hlk]$ id
uid=6000(hlk) gid=20(staff) groups=20(staff),
0(wheel), 80(admin), 160(cvs)
[hlk@fischer hlk]$

[root@fischer hlk]# id
uid=0(root) gid=0(wheel) groups=0(wheel), 1(daemon),
2(kmem), 3(sys), 4(tty), 5(operator), 20(staff),
31(guest), 80(admin)
[root@fischer hlk]#
\end{alltt}

\begin{list1}
\item When showing a dollar sign you are logged in as a regular user
\item while a hash mark means you are the root - super user, no restrictions
\end{list1}






\slide{Manual System}

\hlkimage{7cm}{images/unix-command-1.pdf}

\begin{quote}
 It is a book about a Spanish guy called Manual. You should read it.
       -- Dilbert
\end{quote}

\begin{list1}
\item The Unix/Linux manual system is where you find the options, commands and file formats
\item Manuals must be installed, if not install them immediately
\item Very similar across Unix variants, OpenBSD is known for having an excellent manual pages
\item \verb+man -k+ allows keyword search similar can be done using \verb+apropos+
\end{list1}

Try \verb+man crontab+ and \verb+man 5 crontab+


\slide{Example Manual Page}

\begin{alltt}\footnotesize
\small
NAME
     cal - displays a calendar
SYNOPSIS
     cal [-jy] [[month]  year]
DESCRIPTION
   cal displays a simple calendar.  If arguments are not specified, the cur-
   rent month is displayed.  The options are as follows:
   -j      Display julian dates (days one-based, numbered from January 1).
   -y      Display a calendar for the current year.

The Gregorian Reformation is assumed to have occurred in 1752 on the 3rd
of September.  By this time, most countries had recognized the reforma-
tion (although a few did not recognize it until the early 1900's.)  Ten
days following that date were eliminated by the reformation, so the cal-
endar for that month is a bit unusual.
\end{alltt}

\slide{The year 1752}

\begin{alltt}\footnotesize
  user@Projects:communication-and-network-security$ cal 1752
...
         April                  May                   June
  Su Mo Tu We Th Fr Sa  Su Mo Tu We Th Fr Sa  Su Mo Tu We Th Fr Sa
            1  2  3  4                  1  2      1  2  3  4  5  6
   5  6  7  8  9 10 11   3  4  5  6  7  8  9   7  8  9 10 11 12 13
  12 13 14 15 16 17 18  10 11 12 13 14 15 16  14 15 16 17 18 19 20
  19 20 21 22 23 24 25  17 18 19 20 21 22 23  21 22 23 24 25 26 27
  26 27 28 29 30        24 25 26 27 28 29 30  28 29 30
                        31
          July                 August              September
  Su Mo Tu We Th Fr Sa  Su Mo Tu We Th Fr Sa  Su Mo Tu We Th Fr Sa
            1  2  3  4                     1  {\bf        1  2 14 15 16}
   5  6  7  8  9 10 11   2  3  4  5  6  7  8  17 18 19 20 21 22 23
  12 13 14 15 16 17 18   9 10 11 12 13 14 15  24 25 26 27 28 29 30
  19 20 21 22 23 24 25  16 17 18 19 20 21 22
  26 27 28 29 30 31     23 24 25 26 27 28 29
                        30 31
...
\end{alltt}

\slide{Linux file system and konfiguration}

.
\hlkrightpic{8cm}{0cm}{unix-vfs.pdf}
\begin{list2}

\item Unix/Linux uses a virtual filesystem\\
\url{https://en.wikipedia.org/wiki/Unix_filesystem}
\item No drive letters, just disks mounted in a common tree
\item Everything starts with the file system root \verb+/+ - forward
\item An important directory is \verb+/etc/+ which includes a lot of configuration for the system and applications
\end{list2}




\slidenext{Buy the books!}



\end{document}
