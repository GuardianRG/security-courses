\documentclass[Screen16to9,17pt]{foils}
\usepackage{zencurity-slides}

\externaldocument{communication-and-network-security-exercises}
\selectlanguage{english}

\begin{document}

\mytitlepage
{8. Network Intrusion Detection}
{Communication and Network Security \the\year}

\slide{Goals for today}

\hlkimage{10cm}{zeek-ids.png}

Todays goals:
\begin{list2}
\item See how sniffing can be automated using two example tools Zeek and Suricata
\item Use Ansible to provision services - which can easily be modified to cover multiple servers
\item See some network tool configs
\end{list2}


\slide{Plan for today}

\hlkimage{4cm}{switch-1.pdf}

\begin{list1}
\item Subjects
\begin{list2}
\item Intrusion Detection Systems
\item NIDS vs HIDS
\item Suricata Zeek
\item Network Security Data Visualization
\item Kibana Dashboards
\end{list2}
\item Exercises
\begin{list2}
\item Run Zeek and Suricata on small pcaps
\end{list2}
\end{list1}

\slide{Time schedule}

\begin{list2}
\item 17:00 - 18:15\\
Introduction and basics
\item 30min break\\

\item 18:45 - 19:30\\

\item 15min break\\

\item 19:45 -20:30 45min\\
\end{list2}


\slide{Reading Summary}

\begin{quote}
ANSM chapter (7,8),9,10 - 140 pages\\
DETECTION MECHANISMS\\
Generally, detection is a function of software that parses through collected data in order to generate alert data. This software is referred to as a detection mechanism.
\end{quote}

\begin{quote}
Chapter 7 Detection Mechanisms, Indicators of Compromise, and Signatures\\
Chapter 8 Reputation-Based Detection\\
Chapter 9 Signature-Based Detection with Snort and Suricata\\{\bf
Chapter 10 The Bro Platform} // Now Zeek
\end{quote}

Zeek in the default configuration activates 10.000s of script lines out-of-the-box.\\
Gives great output with little effort and complements Suricata/NIDS



\slide{Reading Summary, continued}

\hlkimage{6cm}{zeek-ids.png}

\begin{list1}
\item ANSM chapter 7: Detection Mechanisms and Indicators of Compromise, and Signatures

\begin{list2}
\item Indicators of Compromise (IOC) any piece of information that can be used to objectively describe a
network intrusion, expressed in a platform-independent manner
\item Background information, useful when we talk about Zeek (previously Bro) later
\item Intrusion Detection Systems try to detect ... but what if we know that some domains, servers, IPs etc are signs of bad activity - even compromise
\item IP reputation - some IPs are used for controlling malware command and control (C2) servers etc.
\item A signature can contain one or more IOCs
\end{list2}
\end{list1}


\slide{Reading Summary, continued}

%\hlkimage{7cm}{NSM_Phases-300x238.png}

\begin{list1}
\item ANSM chapter 8: Reputation-Based Detection
\begin{list2}
\item The most basic form of intrusion detection is reputation-based detection
\item Similar concept to blacklisting SMTP spam relays
\item I often recommend \link{https://github.com/stamparm/maltrail} as a source of lists
\item Other sources are lists like RIPE NCC delegated, which IP prefixes are handed out in different countries\\
\link{https://ftp.ripe.net/pub/stats/ripencc/delegated-ripencc-extended-latest}\\
\verb+ripencc|DK|ipv4|185.129.60.0|1024|20151130|allocated|+
\item Mentions SiLK \link{https://tools.netsa.cert.org/silk/}\\
If we end up having time today, or another day, we should look into this tool chain also!
\end{list2}
\end{list1}



\slide{Reading Summary, continued}

\hlkimage{7cm}{suricata.png}

\begin{list1}
\item ANSM chapter 9: Signature-Based Detection with Snort and Suricata
\begin{list2}
\item Suricata IDS
\item IDS rules are introduced
\item I recommend a commercial license for the EmergingThreats ruleset
\end{list2}
\end{list1}

\slide{Reading Summary, continued}

\hlkimage{15cm}{zeek-overview.png}

\begin{list1}
\item ANSM chapter 10: The Zeek (Bro) Platform
\begin{list2}
\item Zeek concepts and logs - many useful ones by default!
\end{list2}
\end{list1}


\slide{Intrusion Detection}

\begin{list2}
\item networkbased intrusion detection systems (NIDS)
\item host based intrusion detection systems (HIDS)
\end{list2}

\slide{Indicators of Compromise and Signatures}

\begin{quote}
An IOC is any piece of information that can be used to objectively describe a
network intrusion, expressed in a platform-independent manner. This could include a simple indicator such as the IP address of a command and control (C2) server or a complex set of behaviors that indicate that a mail server is being used as a malicious SMTP relay.

When an IOC is taken and used in a platform-specific language or format, such as a Snort Rule or a Bro-formatted file, it becomes part of a signature. A signature can contain one or more IOCs.
\end{quote}

Source: Applied Network Security Monitoring Collection, Detection, and Analysis, 2014 Chris Sanders

\slide{Reading Summary, False Positives}

\begin{list2}
\item True Positive (TP). An alert that has correctly identified a specific activity. If a signature was designed to detect a certain type of malware, and an alert is generated when that malware is launched on a system, this would be a true positive, which is what we strive for with every deployed signature.Indicators of Compromise and Signatures
\item False Positive (FP). An alert has incorrectly identified a specific activity. If a signature was designed to detect a specific type of malware, and an alert is generated for an instance in which that malware was not present, this would be a false positive.
\item True Negative (TN). An alert has correctly not been generated when a specific activity has not occurred. If a signature was designed to detect a certain type of malware, and no alert is generated without that malware being launched, then this is a true negative, which is also desirable. This is difficult, if not impossible, to quantify in terms of NSM detection.
\item False Negative (FN). An alert has incorrectly not been generated when a specific activity has occurred.
\end{list2}

Source: Applied Network Security Monitoring Collection, Detection, and Analysis, 2014 Chris Sanders

\slide{Network Security Monitoring}

\hlkimage{4cm}{network-security-monitoring.png}

\begin{list1}
\item Network Security Monitoring (NSM) - monitoring networks for intrusions, and reacting to those
\item Recommend the book \emph{The Practice of Network Security Monitoring
Understanding Incident Detection and Response}
by Richard Bejtlich
July 2013
\item Example systems are Security Onion \link{https://securityonion.net/} or\\ SELKS \link{https://www.stamus-networks.com/open-source/}
\end{list1}



\slide{The Zeek Network Security Monitor}

\hlkimage{14cm}{zeek-overview.png}

The Zeek Network Security Monitor is not a single tool, more of a
powerful network analysis framework

Zeek is the tool formerly known as Bro, changed name in 2018. \link{https://www.zeek.org/}


\slide{Zeek IDS is}

\hlkimage{14cm}{zeek-ids.png}

\begin{quote}
While focusing on network security monitoring, Zeek provides a comprehensive platform for more general network traffic analysis as well. Well grounded in more than 15 years of research, Zeek has successfully bridged the traditional gap between academia and operations since its inception.
\end{quote}

\link{https://www.Zeek.org/}


\slide{Suricata IDS/IPS/NSM}
\hlkimage{6cm}{suricata.png}

\begin{quote}
Suricata is a high performance Network IDS, IPS and Network Security Monitoring engine.
\end{quote}

 \link{http://suricata-ids.org/}
 \link{http://openinfosecfoundation.org}

{\bf We will now move to the workshop materials:}\\
Suricata, Zeek og DNS Capture\\
{\small\link{https://github.com/kramse/security-courses/tree/master/courses/networking/suricatazeek-workshop}}



\slidenext

\end{document}
