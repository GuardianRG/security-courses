\documentclass[Screen16to9,17pt]{foils}
\usepackage{zencurity-slides}

\externaldocument{communication-and-network-security-exercises}
\selectlanguage{english}

\begin{document}

\mytitlepage
{9. Network Forensics}
{Communication and Network Security \the\year}


\slide{Goals for today}

\hlkimage{10cm}{kibana-suricata.png}

Todays goals:
\begin{list2}
\item Talk about centralized logging using syslog as example
\item We already used logstash and Kibana
\item See and develop small pattern matching Grok expressions
\end{list2}


\slide{Plan for today}

\begin{list1}
\item Subjects
\begin{list2}
\item Centralized syslog
\item Netflow data
\item Collect Network Evidence
\item Analyze Network data
\item Network Forensics
\item (Create Incident Reports)
\end{list2}
\item Exercises
\begin{list2}
\item Create a Kibana dashboard for looking at logs
\item (Run forensics similar to ENISA examples)

\end{list2}
\end{list1}

\slide{Reading Summary}

\begin{list1}
\item ANSM chapter 4,(5,6 moved) - 24 pages
\item 4. Session Data
\item 5. Full Packet Capture
\item 6. Packet String Data
\end{list1}


\slide{Reading Summary, continued}

\hlkimage{6cm}{zeek-ids.png}

\begin{list1}
\item ANSM chapter 4: Session Data
\begin{list2}
\item Netflow and IPFIX - described on more detail
\item Introduce and mentions another tool SiLK \link{https://tools.netsa.cert.org/silk/}\\
If we end up having time today, or another day, we should look into this tool chain also!
\end{list2}
\end{list1}


\slide{IP reputation}

\begin{list1}
\item Zeek documentation Intel framework\\
\link{https://docs.zeek.org/en/stable/frameworks/intel.html}\\
\item Suricata reputation support\\
\link{https://suricata.readthedocs.io/en/suricata-4.0.5/reputation/index.html}
\end{list1}

\exercise{ex:zeekioc}

\slide{Logging today}
\hlkimage{8cm}{moloch-sessions}

\begin{list1}
\item Log analysis is required today - and we have many logs
\item Gather logs, parse logs, explain logs - fix stuff
\item Search your logs with the Elastic stack
\item Show sample logs from Suricata, Sudo, SSH, ... what we have
\end{list1}


\slide{How to get started}

\begin{list1}
\item How to get started searching for security events?
\item Collect basic data from your devices and networks
\begin{list2}
\item Netflow data from routers
\item Session data from firewalls
\item Logging from applications: email, web, proxy systems
\end{list2}
\item {\bf Centralize!}
\item Process data
\begin{list2}
\item Top 10: interesting due to high frequency, occurs often, brute-force attacks
\item {\it ignore}
\item Bottom 10: least-frequent messages are interesting
\end{list2}
\end{list1}


\slide{Centralized syslog}

\begin{list1}
\item Logfiler er en nødvendighed for at have et transaktionsspor
\item Logfiler giver mulighed for statistik
\item Logfiler er desuden nødvendige for at fejlfinde
\item Det kan være relevant at sammenholde logfiler fra:
\begin{list2}
\item routere
\item firewalls
\item webservere
\item intrusion detection systemer
\item adgangskontrolsystemer
\item ...
\end{list2}
\item Husk - tiden er vigtig! Network Time Protocol (NTP) anbefales
\item Husk at logfilerne typisk kan slettes af en angriber -
  hvis denne får kontrol med systemet
\end{list1}

\slide{syslog}

\begin{list1}
\item syslog er system loggen på UNIX og den er effektiv
  \begin{list2}
\item man kan definere hvad man vil se og hvor man vil have det
  dirigeret hen
\item man kan samle det i en fil eller opdele alt efter programmer og
  andre kriterier
\item man kan ligeledes bruge named pipes - dvs filer i filsystemet
  som tunneller fra chroot'ed services til syslog i det centrale system!
\item man kan nemt sende data til andre systemer
  \end{list2}
\item Man bør lave en centraliseret løsning
\end{list1}

\slide{syslogd.conf eksempel}
\begin{alltt}
\small
*.err;kern.debug;auth.notice;authpriv.none;mail.crit    /dev/console
*.notice;auth,authpriv,cron,ftp,kern,lpr,mail,user.none /var/log/messages
kern.debug;user.info;syslog.info                        /var/log/messages
auth.info                                               /var/log/authlog
authpriv.debug                                          /var/log/secure
...
# Uncomment to log to a central host named "loghost".
#*.notice;auth,authpriv,cron,ftp,kern,lpr,mail,user.none        @loghost
#kern.debug,user.info,syslog.info                               @loghost
#auth.info,authpriv.debug,daemon.info                           @loghost
\end{alltt}

\slide{Andre syslogs syslog-ng}

\begin{list2}
\item der findes andre syslog systemer eksempelvis syslog-ng
\item konfigureres gennem \verb+/etc/syslog-ng/syslog-ng.conf+
\end{list2}

\begin{alltt}
\small
options \{
        long_hostnames(off);
        sync(0);
        stats(43200);
\};

source src { unix-stream("/dev/log"); internal(); pipe("/proc/kmsg"); };
destination messages { file("/var/log/messages"); };
destination console_all { file("/dev/console"); };
log { source(src); destination(messages); };
log { source(src); destination(console_all); };
\end{alltt}
Kan eksempelvis TCP og garanteret aflevering af beskeder

\slide{Web server access log}

\begin{alltt}
\footnotesize
root# tail -f access_log
::1 - - [19/Feb/2004:09:05:33 +0100] "GET /images/IPv6ready.png
HTTP/1.1" 304 0
::1 - - [19/Feb/2004:09:05:33 +0100] "GET /images/valid-html401.png
HTTP/1.1" 304 0
::1 - - [19/Feb/2004:09:05:33 +0100] "GET /images/snowflake1.png
HTTP/1.1" 304 0
::1 - - [19/Feb/2004:09:05:33 +0100] "GET /~hlk/security6.net/images/logo-1.png
HTTP/1.1" 304 0
2001:1448:81:beef:20a:95ff:fef5:34df - - [19/Feb/2004:09:57:35 +0100]
"GET / HTTP/1.1" 200 1456
2001:1448:81:beef:20a:95ff:fef5:34df - - [19/Feb/2004:09:57:35 +0100]
"GET /apache_pb.gif HTTP/1.1" 200 2326
2001:1448:81:beef:20a:95ff:fef5:34df - - [19/Feb/2004:09:57:36 +0100]
"GET /favicon.ico HTTP/1.1" 404 209
2001:1448:81:beef:20a:95ff:fef5:34df - - [19/Feb/2004:09:57:36 +0100]
"GET /favicon.ico HTTP/1.1" 404 209
\end{alltt}
\vskip 1cm

Web server logs are pretty standardized, common log format.

\exercise{ex:syslogd-basic}


\slide{Big Data tools: Elasticsearch}


{\color{green}\Large elasticsearch}

Book: the definitive guide

{\small
\link{http://www.elasticsearch.org/guide/en/elasticsearch/guide/current/index.html}

\link{http://www.elasticsearch.org/overview/kibana/}

\link{http://www.elasticsearch.org/overview/logstash/}
}

\vskip 1cm
\centerline{We are all Devops now, even security people!}


\slide{Ansible configuration management}

\begin{alltt}\small
- apt: name={{ item }} state=latest
  with_items:
        - unzip
        - elasticsearch
        - logstash
        - redis-server
        - nginx
- lineinfile: "dest=/etc/elasticsearch/elasticsearch.yml state=present
  regexp='script.disable_dynamic: true' line='script.disable_dynamic: true'"
- lineinfile: "dest=/etc/elasticsearch/elasticsearch.yml state=present
  regexp='network.host: localhost' line='network.host: localhost'"
- name: Move elasticsearch data into /data
  command: creates=/data/elasticsearch mv /var/lib/elasticsearch /data/
- name: Make link to /data/elasticsearch
  file: state=link src=/data/elasticsearch path=/var/lib/elasticsearch
\end{alltt}
\vskip 5mm
\centerline{only requires SSH+python \link{http://www.ansible.com}}


% Suricata, Logstash, Elasticsearch, D3JShttp://d3js.org/
\slide{Kibana}

\hlkimage{12cm}{kibanascreenshothomepagebannerbigger.jpg}

\centerline{Highly recommended for a lot of data visualisation}

Non-programmers can create, save, and share dashboards

Source:
\link{https://www.elastic.co/products/kibana}



\slide{Logstash pipeline }

\begin{quote}
  Logstash is an open source, server-side data processing pipeline that ingests data from a multitude of sources simultaneously, transforms it, and then sends it to your favorite “stash.” (Ours is Elasticsearch, naturally.)\\
  \link{https://www.elastic.co/products/logstash}
\end{quote}

\begin{verbatim}
input { stdin { } }
output {
  elasticsearch { host => localhost }
  stdout { codec => rubydebug }
}
\end{verbatim}



\begin{list2}
\item Logstash receives via {\bf input}
\item Processes with {\bf filters} - grok
\item Forward events with {\bf output}
\end{list2}

%Source:
%Config snippet from recommended link\\
%{\small\link{http://logstash.net/docs/1.4.1/tutorials/getting-started-with-logstash}}


\slide{Logstash as SNMPtrap and syslog server}

{\footnotesize
\begin{verbatim}
input {
  snmptrap {
    host => "0.0.0.0"
    type => "snmptrap"
    port => 1062
    community => "xxxxx"   }
  tcp {
    port => 5000
    type => syslog  }
  udp {
    port => 5000
    type => syslog  }
}
\end{verbatim}
}

\begin{list2}
\item We run logstash on port 5000 - but use IPtables port forwarding
\item Have you even configured SNMP traps?
\item Maybe you have a device sending SNMP traps right now ...
\end{list2}

\slide{IPtables forwarding}

{\footnotesize
\begin{verbatim}
*nat
:PREROUTING ACCEPT [0:0]
# redirect all incoming requests on port 514 to port 5000
-A PREROUTING -p tcp --dport 514 -j REDIRECT --to-port 5000
-A PREROUTING -p udp --dport 514 -j REDIRECT --to-port 5000
-A PREROUTING -p udp --dport 162 -j REDIRECT --to-port 1062
COMMIT
\end{verbatim}
}

\centerline{Inserted near beginning of /etc/ufw/before.rules on Ubuntu}

Remember defense in depth, dont run a privileged Java VM process as root \smiley

\slide{Grok expresssions}

{\footnotesize
\begin{verbatim}
  filter {
    if [type] == "syslog" {
      grok {
        match => { "message" => "%{SYSLOGTIMESTAMP:syslog_timestamp}
        %{SYSLOGHOST:syslog_hostname} %{DATA:syslog_program}
        (?:\[%{POSINT:syslog_pid}\])?: %{GREEDYDATA:syslog_message}" }
        add_field => [ "received_at", "%{@timestamp}" ]
        add_field => [ "received_from", "%{host}" ]
      }
      syslog_pri { }
      date {
        match => [ "syslog_timestamp", "MMM  d HH:mm:ss", "MMM dd HH:mm:ss" ]
      }
    }
  }
\end{verbatim}
}

\begin{list2}
\item Logstash filter expressions grok can normalize and split data into fields
\end{list2}

Source:
Config snippet from recommended link\\
{\small\link{http://logstash.net/docs/1.4.1/tutorials/getting-started-with-logstash}}


\slide{Grok expresssions, sample from my archive}

{\footnotesize
\begin{verbatim}
filter {
# decode some SSHD
if [syslog_program] == "sshd" {
  grok {
# May 20 10:27:08 odn1-nsm-01 sshd[4554]: Accepted publickey for hlk from
10.50.11.17 port 50365 ssh2: DSA 9e:fd:3b:3d:fc:11:0e:b9:bd:22:71:a9:36:d8:06:c7

match => { "message" => "%{SYSLOGTIMESTAMP:timestamp} %{HOSTNAME:host_target}
sshd\[%{BASE10NUM}\]: Accepted publickey for %{USERNAME:username} from
  %{IP:src_ip} port %{BASE10NUM:port} ssh2" }

# "May 20 10:27:08 odn1-nsm-01 sshd[4554]: pam_unix(sshd:session):
session opened for user hlk by (uid=0)"
match => { "message" => "%{SYSLOGTIMESTAMP:timestamp} %{HOSTNAME:host_target}
sshd\[%{BASE10NUM}\]: pam_unix\(sshd:session\): session opened for user
%{USERNAME:username}" }
\end{verbatim}
}

\begin{list2}
\item Logstash filter expressions grok can normalize and split data into fields
\end{list2}



\slide{View data efficiently}

\hlkimage{10cm}{logstash-search.png}

\begin{list1}
\item View data by digging into it easily - must be fast
\item Logstash and Kibana are just examples, but use indexing to make it fast!
\item Other popular examples include Graylog and Grafana
\end{list1}


\slide{Suricata with Dashboards}

\hlkimage{12cm}{kibana-suricata.png}

Picture from Twitter\\
\link{https://twitter.com/nullthreat/status/445969209840128000}\\

\link{http://suricata-ids.org/}

\exercise{ex:kibana-dashboard}


\slide{Collect Network Evidence from the network}

\begin{list1}
\item Network Flows
\item Cisco standard NetFlow version 5 defines a flow as a unidirectional sequence of packets that all share the following 7 values:
\begin{list2}
\item Ingress interface (SNMP ifIndex)
\item Source IP address
\item Destination IP address
\item IP protocol
\item Source port for UDP or TCP, 0 for other protocols
\item Destination port for UDP or TCP, type and code for ICMP, or 0 for other protocols
\item IP Type of Service
\end{list2}
\end{list1}

Source: \link{https://en.wikipedia.org/wiki/NetFlow}


\slide{Netflow}

\begin{slidelist}
\item Netflow is getting more important, more data share the same links
\item Accounting is important
\item Detecting DoS/DDoS and problems is essential
\item Netflow sampling is vital information - 123Mbit, but what kind of traffic
\item NFSen is an old but free application
\link{http://nfsen.sourceforge.net/}
\item Currently also investigating sFlow - hopefully more fine grained
\item sFlow, short for "sampled flow", is an industry standard for packet export at Layer 2 of the OSI model, \\
\link{https://en.wikipedia.org/wiki/SFlow}
\end{slidelist}

\slide{Netflow using NFSen}

\hlkimage{13cm}{images/nfsen-overview.png}


\slide{ Netflow NFSen}

\hlkimage{17cm}{nfsen-udp-flood.png}

\centerline{An extra 100k packets per second from this netflow source (source is a router)}

\slide{Netflow processing from the web interface}

\hlkimage{12cm}{images/nfsen-processing-1.png}

\centerline{Bringing the power of the command line forward}


\slide{Next steps}

Always improving things:
\begin{list1}
\item Suricata IDS \link{http://www.openinfosecfoundation.org/}
\item More graphs, with {\bf automatic identification} of IPs under attack
\item Identification of {\bf short sessions without data} - spoofed addresses
\item Alerting from {\bf existing} devices
\item Dashboards with key measurements
\end{list1}

\vskip 2cm
\centerline{\bf\Large Conclusion: Combine tools!}


\slide{Network Security Through Data Analysis}

\hlkimage{4cm}{network-security-through-data-analysis.png}

Low page count, but high value! Recommended.

Network Security through Data Analysis, 2nd Edition
By Michael S Collins
Publisher: O'Reilly Media
2015-05-01: Second release, 348 Pages

New Release Date: August 2017

And we have the ANSM book

\slide{Network tools - more examples}

\hlkimage{12cm}{using-packetq.png}

\begin{list2}
\item DNS: DSC and PacketQ \link{https://github.com/DNS-OARC/PacketQ}
\item Packetbeat \link{https://www.elastic.co/products/beats/packetbeat}

\item \link{http://securityblog.switch.ch/2013/01/22/using-packetq/}
\item \link{http://jpmens.net/2013/05/27/server-agnostic-logging-of-dns-queries-responses/}
\end{list2}





\slide{Network Forensics ENISA}

\begin{quote}
  The European Union Agency for Network and Information Security (ENISA) is a centre of expertise for cyber security in Europe.

ENISA is contributing to a high level of network and information security (NIS) within the European Union, by developing and promoting a culture of NIS in society to assist in the proper functioning of the internal market.
\end{quote}

\link{https://www.enisa.europa.eu/}

ENISA has published a number of network forensics documents which are free to use, so these are our basics.

\slide{Forensic analysis}

\begin{quote}
Network forensics is a sub-branch of digital forensics relating to the monitoring and analysis of computer
network traffic for the purposes of information gathering, legal evidence, or intrusion detection 5 .
\end{quote}

\begin{list1}
\item Systems used to collect network data for forensics use usually come in three forms:
\begin{list2}
\item Packet capture: All packets passing through a certain traffic point are captured and written to storage
\item Intrusion detection systems
\item Network flow sensors
\end{list2}
\end{list1}

The acronym OSCAR 8 stands for: Obtain information,
Strategize,
Collect evidence,
Analyse,
Report

Source: Forensic analysis Network Incident Response Handbook, Document for teachers 1.0 DECEMBER 2016, ENISA\\
\verb+EXE2_Forensic_analysis_II-Handbook.pdf+




\slide{ENISA exercises}

\begin{list2}
\item We will use these as examples:
\item ENISA Presenting, correlating and filtering various feeds Handbook, Document for teachers\\ \link{https://www.enisa.europa.eu/topics/trainings-for-cybersecurity-specialists/online-training-material/documents/presenting-correlating-and-filtering-various-feeds-handbook}
\item ENISA Forensic analysis, Network Incident Response\\ \link{https://www.enisa.europa.eu/topics/trainings-for-cybersecurity-specialists/online-training-material/documents/2016-resources/exe2_forensic_analysis_ii-handbook}
\item ENISA Network Forensics, Handbook, Document for teachers\\ \link{https://www.enisa.europa.eu/topics/trainings-for-cybersecurity-specialists/online-training-material/documents/network-forensics-handbook}
\end{list2}

%\slide{Create Incident Reports}


\slidenext

\end{document}
